\section{Conclusion} \label{sec:conclusion}

Downloading resources over the internet is indeed a risky endeavor. Resource
integrity and other Supply Chain Attacks are becoming more frequent and their
impact more widely felt. In this work, we showed that the de facto standard for
protecting the integrity of arbitrary resources on the internet---the use of
\emph{checksums}---is insufficient and often ineffective. We presented
\SYSTEM{}, a practical resource verification protocol that automates the tedious
parts of checksum verification while leveraging pre-existing high availability
systems to ensure resources and their checksums are not vulnerable to
co-hosting. Further, we demonstrated the effectiveness and practicality of our
approach versus real-world resource integrity attacks in a production
application.

The results of our evaluation show that our approach is more effective than
checksums and prior work mitigating integrity attacks for arbitrary resources on
the internet. Further, we show \SYSTEM{} is capable of guarding against a
variety of attacks, is deployable at scale for providers that already maintain a
DNS presence, and can be deployed without fear of adversely affecting the user
experience of clients that are not \SYSTEM{}-aware.

Though not a panacea, we believe our protocol significantly raises the bar for
the attacker. We intend to continue developing our extension and we make it
available to a wide audience (see \secref{availability}).
