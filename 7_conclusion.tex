\section{Conclusion} \label{sec:conclusion}

Downloading resources over the internet is indeed a risky endeavour. Resource
integrity attacks, and Supply Chain Attacks more broadly, are becoming more
frequent and their impact more widely felt. This paper shows that the de facto
standard for addressing resource integrity risk---the use of \emph{checksums}
coupled with a secure transport layer---is an insufficient and often ineffective
solution. We proposed a novel resource validation scheme meant as a complete
replacement for checksum based approaches: \SYSTEM{}, which automates the
tedious parts of verification to eliminate user apathy while leveraging
highly-available authenticated distributed systems to ensure resources and
checksums are not co-hosted. Further, we demonstrate the effectiveness and
practicality of our approach versus resource integrity attacks in a real-world
system.

The results of our evaluation show that \SYSTEM{} is more effective than
checksums and other existing solutions at mitigating resource integrity attacks
against arbitrary resources on the internet. Further, we show \SYSTEM{} detects
a wide variety of real-world integrity errors across a diverse set of platforms.
\SYSTEM{} is both scalable and immediately deployable for organizations that
secure their DNS zone(s) with DNSSEC.

We make our DNS implementation of \SYSTEM{} available open source so that others
can extend it or compare to it. Our hope is that this work motivates further
exploration of resource integrity Supply Chain Attack mitigation
methods.~\footnote{The \SYSTEM{} Chrome extension is available at
https://tinyurl.com/dnschk-actual}.
