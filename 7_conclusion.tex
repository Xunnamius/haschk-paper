\section{Conclusion} \label{sec:conclusion}

\TODO{Rewrite this section}

Downloading resources over the internet is indeed a risky endeavor. Resource
integrity attacks, and Supply Chain Attacks more broadly, are becoming more
frequent and their impact more widely felt. This paper shows that the de facto
standard for addressing resource integrity risk---the use of \emph{checksums}
coupled with a secure transport layer---is an insufficient and often ineffective
solution. We propose a novel practical resource verification approach meant as a
complete replacement for checksum based approaches: \SYSTEM{}, which automates
the tedious parts of verification to eliminate user apathy while leveraging high
availability systems to ensure resources and checksums are not co-hosted.
Further, we demonstrate the effectiveness and practicality of our approach
versus resource integrity attacks in a real-world system.

The results of our evaluation show that our approach is more effective than
checksums and other attempts at mitigating resource integrity attacks for
arbitrary resources on the internet. Further, we show \SYSTEM{} is capable of
detecting a wide variety of real-world integrity errors, significantly raising
the bar for the attacker. \SYSTEM{} is deployable at scale for providers that
already maintain a web presence; this can be done without fear of adversely
affecting the user experience of non-compliant clients.
