\section{Related Work} \label{sec:related}

In this section, we examine prior approaches to guaranteeing resource integrity
over the internet and other related work. We also highlight the drawbacks to
these approaches and how our protocol overcomes them. \\

\noindent\textbf{Anti-malware software, heuristics, and blacklists.}
Anti-malware software is a heuristic-based program designed for the specific
purpose of detecting and removing various kinds of malware. However, updates to
anti-malware definitions often lag behind or occur in response to the release of
crippling malware. For example, during the 2017 compromise of the HandBrake
distribution mirror, users who first ran the compromised HandBrake image through
\textit{VirusTotal}---a web service that will run a resource through several
dozen popular anti-malware products---received a report claiming no infections
were detected despite the presence of the Proton malware~\cite{SCA-HB1}. In the
2012 compromise of SourceForge's CDN, the malicious changes to the phpmyAdmin
image don't appear as malware to anti-malware software~\cite{SCA-PMA1}.

Similarly, all modern browsers employ heuristic and blacklist-based detection
and prevention schemes in an attempt to protect users from malicious content on
the internet. The warnings generated by browser-based heuristics and blacklists
are also reactive rather than proactive; hence, they are generally ineffective
at detecting active or novel attacks on the integrity of the resources
downloaded over the internet.

On the other hand, \SYSTEM{} relies on no heuristics or blacklists and is not
anti-malware software. \SYSTEM{} is a protocol for automating checksum
verification of resources. This insures download integrity---that a user is
receiving the expected resource a provider is advertising---not that the
expected resource is not malware. \\

\noindent\textbf{Link Fingerprints and Subresource Integrity.} The Link
Fingerprints (LF) draft describes an early HTML anchor and URL based resource
integrity verification scheme that ``provides a backward-compatible technique
for resource providers to ensure that the resource originally referenced is the
same as the resource retrieved by an end user.''~\cite{LF}. The World Wide Web
Consortium's (W3C) Subresource Integrity (SRI) describes a similar HTML-based
scheme designed with CDNs and web assets in mind.

Like \SYSTEM{}, both LF and SRI employ cryptographic digests to ensure no
changes of any kind have been made to a resource~\cite{SRI}. Unlike \SYSTEM{},
LF and SRI apply only to resources referenced by script and link HTML elements;
\SYSTEM{} can ensure the integrity of any arbitrary resource downloaded over the
internet. Further, the checksums contained in the HTML source must be accurate
for SRI to work. If the system behind the CDN is compromised, rather than the
CDN itself, the attacker can alter the HTML and inject a malicious checksum to
match a compromised resource, both of which would be dutifully distributed by
the CDN. With \SYSTEM{}, however, an attacker would additionally have to
compromise the distinct backend that advertises the provider's resources, thus
raising the bar. \\

\noindent\textbf{Content-MD5 Header.} The Content-MD5 header field is a
deprecated email and HTTP header that delivers a checksum similar to those used
by Subresource Integrity. It was removed from the HTTP/1.1 specification because
of the inconsistent implementation of partial response handling between
vendors~\cite{HTTP1.1}. Further, the header could be easily stripped off or
modified by proxies and other intermediaries~\cite{MD5Header}. \SYSTEM{}
exhibits none of these weaknesses. \\

\noindent\textbf{Deterministic Build Systems and Binary Transparency.} A
deterministic build system is one that, when given the same source, will
deterministically output the same binary on every run. For example, many
packages in Debian~\cite{ReproBuildsDebian} and Arch Linux can be rebuilt from
source to yield an identical byte for byte result~\cite{ReproBuilds}, allowing
for verification of the \emph{Integration} and perhaps \emph{Development} supply
chain phases (see \tblref{attacks}). Further, using a merkle
tree~\cite{MerkleTree} or similar construction, an additional chain of trust can
be established that allows for public verification of the \emph{Deployment},
\emph{Maintenance}, and \emph{Retirement} supply chain phases. Companies such as
Mozilla refer to the latter as ``Binary Transparency.''

Like \SYSTEM{}, binary transparency establishes a public verification scheme
that allows third party consumers access to a listing of source updates
advertised by a provider~\cite{BinaryTransparency}. Consumers can take advantage
of deterministic build systems and Binary Transparency together to ensure their
software is the same software deployed to every other system. Unlike \SYSTEM{},
binary transparency only allows a user to verify the integrity of source updates
to \emph{binaries}; our protocol allows a user to verify the integrity of
\emph{any arbitrary internet resource} while specifically addressing co-hosting.
\\

\noindent\textbf{Stickler and Cherubini et al.} Stickler~\cite{Stickler} by Levy
et al. is a JavaScript-based stand-in for Subresource Integrity for protecting
the integrity of web application files hosted on CDNs. Unlike \SYSTEM{},
Stickler does not require any modifications to the client to protect resources.
However, as it was designed for CDNs, concerns like co-hosting are outside of
Stickler's threat model. If the content server behind the CDN is compromised, as
was the case with Linux Mint, it would be trivial to modify the manifest file to
inject corrupted checksums.

Similarly, the automated checksum verification approach by Cherubini et
al.~\cite{Cherubini}---also based on Subresource Integrity---is vulnerable to
co-hosting. Cherubini's browser extension works by looking for embedded
checksums in download links and extracting hexadecimal strings that look like
checksums directly from the page. An attacker, after compromising the resource
file, need only modify the page's HTML file to inject a corrupted checksum
matching that resource, causing Cherubini's extension to misreport the
compromised download as safe~\cite{Cherubini}.

Unlike prior work, \SYSTEM{} is both automated and capable of mitigating the
co-hosting threat, significantly raising the bar for the attacker.
