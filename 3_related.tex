\section{Related Work} \label{sec:related}

In this section, we examine current and previous DNS-based and other approaches
to problems related to \SYSTEM{}. We specifically note limiting human factors as
they apply to the application of checksums for resource integrity validation and
how our \DNSSYS{} implementation avoids them.

\noindent\textbf{Anti-Malware Software.}

Anti-malware software are heuristic-based programs designed for the specific
purpose of detecting and removing various kinds of malware. However, updates to
anti-malware definitions often lag behind or occur in response to the release of
crippling malware. For example, during the 2017 compromise of the HandBrake
distribution mirror, users who first ran the compromised HandBrake image through
\textit{VirusTotal}---a web service that will run a resource through several
dozen popular anti-malware products---received a report claiming no infections
were detected, despite the empirically verifiable presence of the Proton
malware.

Worse, not all resource compromises end up looking like malware. In the 2012
compromise of SourceForge's CDN, where a malicious version of phpMyAdmin was
delivered to hundreds of users, the PHP source itself was altered to enable
remote code execution. However, the extraneous code was virtually
indistinguishable from the rest of the raw PHP source in the phpmyAdmin image
being distributed.

At the time of writing (2018), VirusTotal correctly identifies the compromised
version of the Handbrake image as malware.

\noindent\textbf{Browser-based Heuristics and Blacklists.}

Modern browsers employ heuristic and blacklist based detection and prevention
schemes in an attempt to protect users from encountering malicious content on
the internet. Implementations include Google Chrome's \textit{Safe Browsing}
feature, Mozilla Firefox's \textit{Phishing/Download Protection}, and Microsoft
Edge's \textit{malware sniffing} Windows Defender browser bundle.

Similar to anti-malware software, browser-based heuristics and blacklists are a
reactive rather than proactive solution; hence, they are ineffective at
shielding users from attacks on the integrity of the resources downloaded over
the internet.

\noindent\textbf{Cryptographic Data in DNS Resource Records.}

The DNS-Based Authentication of Named Entities (DANE) specification~\cite{DANE1,
DANE2, DANE3} defines the ``TLSA'' and ``OPENPGPKEY'' DNS resource records to
store cryptographic data. These resource record types, along with
``CERT''~\cite{CERT}, ``IPSECKEY''~\cite{IPSECKEY}, those defined by DNS
Security Extensions (DNSSEC)~\cite{DNSSEC}, and others demonstrate that storing
useful cryptographic data retrievable through the DNS network is feasible at
scale. Due the unique requirements of \DNSSYS{}, however, we use ``TXT'' records
to map Resource Identifiers to Authoritative Checksums. In accordance with RFC
5507~\cite{RFC5507}, a production \DNSSYS{} implementation would necessitate the
creation of a new DNS resource record type as no current resource record type
meets the requirements of \DNSSYS{}. \\

\noindent\textbf{PGP/OpenPGP.} Though PGP addresses a fundamentally different
authentication-focused threat model compared with \SYSTEM{}, it is useful to
note: many of the same human and UX factors that make the cryptographically
solid OpenPGP standard and its various implementations so unpleasant for end
users also exist in the context of download integrity verification and
checksums. End users cannot and \textit{will not} be burdened with manually
verifying a checksum; as was the case with PGP 5.0~\cite{PGPBad}, some users are
likely confused by the very notion and function of a checksum, if they are aware
that checksums exist at all. If PGP's adoption issues are any indication, users
of a security solution that significantly complicates an otherwise simple task
are more likely to bypass said solution rather than be burdened with it. To
assume otherwise can have disastrous consequences~\cite{PGPBad} (cf.
\secref{background}). \\

\noindent\textbf{Link Fingerprints and Subresource Integrity.} The Link
Fingerprints (LF) draft describes an early HTML anchor and URL based resource
integrity verification scheme~\cite{LF}. Subresource Integrity (SRI) describes a
similar production-ready HTML-based scheme designed with CDNs and web assets
(rather than generic resources) in mind. Like \SYSTEM{}, both LF and SRI employ
cryptographic digests to ensure no changes of any kind have been made to a
resource~\cite{SRI}. Unlike \SYSTEM{}, LF and SRI rely on the server that hosts
the HTML source to be secure; specifically, the checksums contained in the HTML
source must be accurate for these schemes to work. An attacker that has control
of the web server can alter the HTML and inject a malicious checksum. With
\SYSTEM{}, however, an attacker would additionally have to compromise whichever
authenticated distributed system hosted the mappings between Resource
Identifiers and Authoritative Checksums. \\

\noindent\textbf{Content-MD5 Header.} The Content-MD5 header field is a
deprecated email and HTTP header that delivers a checksum similar to those used
by Subresource Integrity. It was removed from the HTTP/1.1 specification because
of the inconsistent implementation of partial response handling between
vendors~\cite{HTTP1.1}. Further, the header could be easily stripped off or
modified by proxies and other intermediaries~\cite{MD5Header}. \\

\noindent\textbf{Reproducible Builds/Deterministic Build Systems.} A
deterministic build system is one that, when given the same source, will
deterministically output the same binary on every run. For example, many
packages in Debian~\cite{ReproBuildsDebian} and Arch Linux can be rebuilt from
source to yield an identical byte for byte result each time via a reproducible
build process~\cite{ReproBuilds}. When a deterministic build system is coupled
with the \SYSTEM{} approach, a chain of trust can be established that links the
\emph{Development} and \emph{Integration} supply chain phases to the
\emph{Deployment}, \emph{Maintenance}, and \emph{Retirement} supply chain phases
(cf. \tblref{attacks}), further raising the bar for the attacker.

\noindent\textbf{Transparency Approaches.}

\TODO{TODO!}

\noindent\textbf{Other Automated Checksum Verification Solutions.}

\TODO{TODO!}
