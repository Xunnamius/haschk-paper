\section{Current Solutions and Related Work} \label{sec:related}

In this section, we examine current approaches to guaranteeing resource
integrity over the internet and other related work. We also discuss where these
approaches come up short. We further highlight some of the limiting factors to
these approaches and how \SYSTEM{} implementations might avoid them.

\subsection{Current Solutions}

Solutions implemented to protect users from compromised resource downloads in
the wild include anti-malware software, browser-based heuristics and blacklists.

Anti-malware software is a heuristic-based program designed for the specific
purpose of detecting and removing various kinds of malware. However, updates to
anti-malware definitions often lag behind or occur in response to the release of
crippling malware. For example, during the 2017 compromise of the HandBrake
distribution mirror, users who first ran the compromised HandBrake image through
\textit{VirusTotal}---a web service that will run a resource through several
dozen popular anti-malware products---received a report claiming no infections
were detected, despite the empirically verifiable presence of the Proton
malware~\cite{SCA-HB1}. At the time of writing, VirusTotal correctly identifies
the compromised version of the Handbrake image as malware.

Worse, not all resource compromises end up looking like malware. In the 2012
compromise of SourceForge's CDN, where a malicious version of phpMyAdmin was
delivered to hundreds of users, the PHP source itself was altered to enable
remote code execution. However, the extraneous code was virtually
indistinguishable from the rest of the raw PHP source in the phpmyAdmin image
being distributed~\cite{SCA-PMA1}.

On the other hand, all modern browsers employ heuristic and blacklist-based
detection and prevention schemes in an attempt to protect users from malicious
content on the internet. Implementations include Google Chrome's \textit{Safe
Browsing} feature, Mozilla Firefox's \textit{Phishing/Download Protection}, and
Microsoft Edge's \textit{malware sniffing} Windows Defender browser bundle.

The warnings generated by browser-based heuristics and blacklists are a reactive
rather than proactive solution; hence, they are generally ineffective at
detecting active attacks on the integrity of the resources downloaded over the
internet. In the cases where problems \emph{are} detected, users are known to
ignore the resulting security warnings depending on an implementation's
UI~\cite{Clickthrough, Modic, Akhawe, ChromeClickThrough}. In this case,
Cherubini et al conclude taking an opt-out approach similar to that of
anti-malware and spam filters is sufficient~\cite{Cherubini}. That is: 1)
clearly, visibly, and simply assert the danger of the download and 2) privilege
security over choice by rejecting the download by default while making it more
difficult to ignore or click through the security warning. Our \SYSTEM{}
implementations were designed with this guidance in mind, with an ideal
implementation able to rely on Google Chrome's dangerous download
UI~\cite{ChromeClickThrough}.

\subsection{Related Work}

\noindent\textbf{PGP/OpenPGP.} Though PGP addresses a fundamentally different
authentication-focused threat model compared with \SYSTEM{}, it is useful to
note: many of the same human and UX factors that make the cryptographically
solid OpenPGP standard and its various implementations so unpleasant for end
users also exist in the context of download integrity verification and
checksums. End users cannot and \textit{will not} be burdened with manually
verifying a checksum; as was the case with PGP 5.0, some users are confused by
the very notion and function of a checksum, if they are aware that checksums
exist at all~\cite{PGPBad, Cherubini}. Hence, users of a security solution that
significantly complicates an otherwise simple task are more likely to bypass
that solution rather than be inconvenienced by it. \\

\noindent\textbf{Link Fingerprints and Subresource Integrity.} The Link
Fingerprints (LF) draft describes an early HTML anchor and URL based resource
integrity verification scheme~\cite{LF}. Subresource Integrity (SRI) describes a
similar production-ready HTML-based scheme designed with CDNs and web assets
(rather than generic resources) in mind. Like \SYSTEM{}, both LF and SRI employ
cryptographic digests to ensure no changes of any kind have been made to a
resource~\cite{SRI}. Unlike \SYSTEM{}, LF and SRI rely on the server that hosts
the HTML source to be secure; specifically, the checksums contained in the HTML
source must be accurate for these schemes to work. An attacker that has control
of the web server can alter the HTML and inject a malicious checksum. With
\SYSTEM{}, however, an attacker would additionally have to compromise whichever
distributed system hosted the mappings between Resource Identifiers and
Authoritative Checksums. \\

\noindent\textbf{Content-MD5 Header.} The Content-MD5 header field is a
deprecated email and HTTP header that delivers a checksum similar to those used
by Subresource Integrity. It was removed from the HTTP/1.1 specification because
of the inconsistent implementation of partial response handling between
vendors~\cite{HTTP1.1}. Further, the header could be easily stripped off or
modified by proxies and other intermediaries~\cite{MD5Header}. \\

\noindent\textbf{Reproducible Builds/Deterministic Build Systems.} A
deterministic build system is one that, when given the same source, will
deterministically output the same binary on every run. For example, many
packages in Debian~\cite{ReproBuildsDebian} and Arch Linux can be rebuilt from
source to yield an identical byte for byte result each time via a reproducible
build process~\cite{ReproBuilds}. When a deterministic build system is coupled
with the \SYSTEM{} approach, a chain of trust can be established that links the
\emph{Development} and \emph{Integration} supply chain phases to the
\emph{Deployment}, \emph{Maintenance}, and \emph{Retirement} supply chain phases
(cf. \tblref{attacks}), further raising the bar for the attacker. \\

\noindent\textbf{Transparency Approaches.} \TODO{TODO!} \\

\noindent\textbf{Stickler} \TODO{TODO!} \\

\noindent\textbf{Cherubini et al.} \TODO{TODO!}
