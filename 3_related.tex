\section{Current Solutions and Related Work} \label{sec:related}

In this section, we examine current approaches to guaranteeing resource
integrity over the internet and other related work. We also discuss where these
approaches come up short. We further highlight some of the limiting factors to
these approaches and how \SYSTEM{} implementations might avoid them.

\subsection{Current Solutions}

Solutions implemented to protect users from compromised resource downloads in
the wild include anti-malware software, browser-based heuristics and blacklists.

Anti-malware software is a heuristic-based program designed for the specific
purpose of detecting and removing various kinds of malware. However, updates to
anti-malware definitions often lag behind or occur in response to the release of
crippling malware. For example, during the 2017 compromise of the HandBrake
distribution mirror, users who first ran the compromised HandBrake image through
\textit{VirusTotal}---a web service that will run a resource through several
dozen popular anti-malware products---received a report claiming no infections
were detected despite the presence of the Proton malware~\cite{SCA-HB1}. In the
2012 compromise of SourceForge's CDN, the malicious changes to the phpmyAdmin
image don't appear as malware to anti-malware software~\cite{SCA-PMA1}.

Similarly, all modern browsers employ heuristic and blacklist-based detection
and prevention schemes in an attempt to protect users from malicious content on
the internet. The warnings generated by browser-based heuristics and blacklists
are also reactive rather than proactive; hence, they are generally
ineffective at detecting active attacks on the integrity of the resources
downloaded over the internet. In the cases where problems \emph{are} detected,
users are known to ignore the resulting security warnings depending on an
implementation's UI~\cite{Clickthrough, Modic, Akhawe, ChromeClickThrough}. In
this case, Cherubini et al conclude taking an opt-out approach similar to that
of anti-malware and spam filters is sufficient~\cite{Cherubini}. That is: 1)
clearly, visibly, and simply assert the danger of the download and 2) privilege
security over choice by rejecting the download by default while making it more
difficult to ignore or click through the security warning. Our \SYSTEM{}
implementations were designed with this guidance in mind, with an ideal
implementation able to rely on Google Chrome's dangerous download
UI~\cite{ChromeClickThrough}.

\subsection{Related Work}

\noindent\textbf{Link Fingerprints and Subresource Integrity.} The Link
Fingerprints (LF) draft describes an early HTML anchor and URL based resource
integrity verification scheme~\cite{LF}. Subresource Integrity (SRI) describes a
similar production-ready HTML-based scheme designed with CDNs and web assets
(rather than generic resources) in mind. Like \SYSTEM{}, both LF and SRI employ
cryptographic digests to ensure no changes of any kind have been made to a
resource~\cite{SRI}. Unlike \SYSTEM{}, LF and SRI rely on the server that hosts
the HTML source to be secure; specifically, the checksums contained in the HTML
source must be accurate for these schemes to work. An attacker that has control
of the web server can alter the HTML and inject a malicious checksum. With
\SYSTEM{}, however, an attacker would additionally have to compromise whichever
distributed system hosted the mappings between Resource Identifiers and
Authoritative Checksums. \\

\noindent\textbf{Content-MD5 Header.} The Content-MD5 header field is a
deprecated email and HTTP header that delivers a checksum similar to those used
by Subresource Integrity. It was removed from the HTTP/1.1 specification because
of the inconsistent implementation of partial response handling between
vendors~\cite{HTTP1.1}. Further, the header could be easily stripped off or
modified by proxies and other intermediaries~\cite{MD5Header}. \\

\noindent\textbf{Deterministic Build Systems and Binary Transparency.} A
deterministic build system is one that, when given the same source, will
deterministically output the same binary on every run. For example, many
packages in Debian~\cite{ReproBuildsDebian} and Arch Linux can be rebuilt from
source to yield an identical byte for byte result each time via a reproducible
build process~\cite{ReproBuilds}. Further, using a merkle tree~\cite{MerkleTree}
or similar construction, a chain of trust can be established that links the
\emph{Development} and \emph{Integration} supply chain phases (deterministic
builds) to the \emph{Deployment}, \emph{Maintenance}, and \emph{Retirement}
supply chain phases (see \tblref{attacks}). Companies such as Mozilla refer to
using merkle trees in this way as ``Binary Transparency,'' a public verification
scheme that allows third parties access to a complete listing of source updates
applied to a piece of software~\cite{BinaryTransparency}. Third parties can use
Binary Transparency to ensure the binary on their system is the same binary
deployed publicly to every other system.

Unlike \SYSTEM{}, Binary Transparency is about using merkle trees to create
chains of trust for binary updates. \SYSTEM{}, on the other hand, creates a
chain of trust for any arbitrary internet resource, including binaries.\\

\noindent\textbf{Stickler and Cherubini et al.} \TODO{TODO!} \\
