\section{Background} \label{sec:background}

In this section, we describe the motivation for \SYSTEM{}, including four case
studies that frame the threat posed by Supply Chain Attacks (SCAs). We then
examine current methods to detect and prevent resource corruption including
checksums, HTTPS, anti-malware, PKI, and others.

\subsection{Supply Chain Attacks on Resource Integrity}

Every year, supply chain attacks and their impacts are being felt more and more
widely. \TODO{(more; use NIST citations)}

\subsection{Motivation: Cases}
\TODO{I liked this section especially.  If you have time, I think you
  could strengthen this section (and the whole paper) by propagating
  the notion of (a) user apathy and (b) co-hosting throughout
  (assuming you are comfortable with that framing).  That will help
  make the paper's message consistent from the abstract through the
  conclusion.  You say these are the two problems, you talk about
  cases studies that exhibited one or both of the problems (listed
  explicitly as ``user apathy'' or ``co-hosting'') and then you talk
  about the Chrome/FileZilla extensions as addressing user apathy with
  the DNSSEC/DHT implementations addressing co-hosting.}

Here we select four historic attacks we believe most effectively articulate the
threat posed by resource integrity SCAs and how \SYSTEM{} might have been used
to more efficiently mitigate fallout. We examine each attack, noting the
critical points of failure in their checksum-based resource security model. \\

\noindent\textbf{Case 1: PhpMyAdmin.} For an unspecified amount of time circa
2012, a compromised download mirror in SourceForge's official HTTPS-protected
CDN was distributing a malicious version of the popular database administration
software phpMyAdmin~\cite{SCA-PMA3}. The administrator of the mirror in question
confirmed the attack was due to a vulnerability not shared by SourceForge's
other mirrors~\cite{SCA-PMA2}.

Attackers mutated the software image, injecting files that would allow any
attacker aware of their existence to remotely execute arbitrary PHP code on the
victim's system~\cite{SCA-PMA1}. SourceForge estimates approximately 400 unique
users downloaded this corrupted version of phpMyAdmin before the mirror was
disconnected from their CDN, potentially allowing attackers access to the
private customer data of any number of organizations~\cite{SCA-PMA2}.

While the attackers were able to penetrate a mirror in SourceForge's CDN, the
official phpMyAdmin website was entirely unaffected; the authoritative checksums
listed on the site's download page were similarly unaffected~\cite{SCA-PMA2}.
Hence, a user who was sufficiently motivated, had sufficient technical knowledge
of checksums and how to calculate them, and was also privy to the location of
the correct checksum for the official phpMyAdmin image \emph{might} have
noticed the discrepancy between the two digests. Clearly, a majority of users do
not meet these criteria. \\

\noindent\textbf{Case 2: Linux Mint.} In 2016, the Linux Mint team discovered an
intrusion into their official HTTPS-protected distribution
server~\cite{SCA-MINT1}. Attackers mutated download links originally pointing to
the Linux Mint 17.3 Cinnamon edition ISO, redirecting unsuspecting users to a
disparate system hosting a custom Mint ISO compiled with the IRC-based Linux
backdoor malware \emph{Tsunami}~\cite{SCA-MINT2}. The attack affected several
hundred of the downloads during that day, with the attackers claiming that a
``few hundred'' Linux Mint installs were explicitly under their control. The
primary motivation behind the intrusion was the construction of a
botnet~\cite{SCA-MINT3}. The authoritative checksum displayed on the official
website was also mutated to corroborate the backdoored ISO~\cite{SCA-MINT3}.

Storing the checksum elsewhere may have prevented mutations on the checksum;
still, as demonstrated by the first case, such an effort is not itself a
solution. Hosting a checksum on a secondary system is not very useful if users
downloading the resource protected by that checksum are not actually
\emph{checking} it against a manual calculation. \\

\noindent\textbf{Case 3: Havex.} As part of a widespread espionage campaign
beginning in 2010, Russian Intelligence Services targeted the industrial control
systems of numerous aviation, national defense, critical infrastructure,
pharmaceutical, petrochemical, and other companies and organizations with the
Havex remote access trojan~\cite{SCA-HAVEX1, SCA-HAVEX2}. The attack was carried
out in phases whereby innocuous software images hosted on disparate
\emph{legitimate} vendor websites were targeted for replacement with versions
infected with the Havex malware~\cite{SCA-HAVEX2}. The goal here, as is the case
with all SCAs, was to infect victims indirectly by having the Havex malware
bundled into opaque software dependencies, \ie a hardware driver or internal
communication application.

It is estimated that Havex successfully impacted hundreds or even thousands of
corporations and organizations---mostly in United States and 
Europe~\cite{SCA-HAVEX2}. The motivation behind the Havex malware was
intelligence exfiltration and espionage~\cite{SCA-HAVEX1}. How many of these
vendors employed checksums and other mitigations as part of their software
release cycle is not well reported, though investigators note said vendors'
distribution mechanisms were insecure~\cite{SCA-HAVEX2}; however, an automated
resource validation method could very plausibly have partially mitigated the
delivery of compromised software to end users. \\

\noindent\textbf{Case 4: HandBrake.} In May of 2017, users of HandBrake, a
popular open source video transcoder for Mac/OSX, were made aware that they may
have downloaded and installed a trojan riding atop their transcoding software.
Attackers breached a HTTPS-protected HandBrake download mirror, replacing the
legitimate software with a version containing a novel variant of the
\emph{Proton} malware~\cite{SCA-HB1}. The number of users potentially affected
is unreported.

The goal of the attack was the exfiltration of victims' sensitive data,
including entire keychains (unlocked), private keys, browser password databases,
1Password/Lastpass vaults, decrypted files, and victims' personal videos and
other media~\cite{SCA-HB1}. The HandBrake developers recommended users perform
manual checksum validation to determine if their installation media was
compromised~\cite{SCA-HB2}.

Despite the attackers mutating the HandBrake binary, the authoritative checksums
listed on the official HandBrake download page were reportedly left
untouched~\cite{SCA-HB2}. Further, the developers of HandBrake store their
authoritative checksums both on their official website and in their official
GitHub repository~\cite{SCA-HB2}. A sufficiently knowledgeable, sufficiently
motivated user \emph{might} have noticed the discrepancy between their
calculated checksum and the authoritative checksum listed on the download page.

Suppose, however, that the attackers \textit{had} managed to mutate the
checksums on the official website. Then there would be a discrepancy between the
authoritative checksums on the official site and the authoritative checksums in
the GitHub repository---that's \emph{if} users are even aware that a second set
of checksums are available at all. On top of technical knowledge, a user in this
confusing situation is then expected to ``do the right thing,'' whatever that
happens to be in this context.

\subsection{Other Detection and Mitigation Methods}

Detecting and/or mitigating resource integrity SCAs and other active attacks is
a non-trivial and well studied problem~\cite{MD5Header, HTTP1.1, HTTPS, SRI, LF,
OpenPGP1, DNSSEC, PKI}. What follows is a brief overview of current and popular
methods to ensure the integrity of resources exchanged over the internet other
than checksums.

\subsubsection{Anti-Malware Software}

Anti-malware software are heuristic-based programs designed for the specific
purpose of detecting and removing various kinds of malware. However, updates to
anti-malware definitions often lag behind or occur in response to the release of
crippling malware. For example, during the 2016 compromise of the HandBrake
distribution mirror, users who first ran the compromised HandBrake image
through \textit{VirusTotal}---a web service that will run a resource through
several dozen popular anti-malware products---received a report claiming no
infections were detected, despite the empirically verifiable presence of the
Proton Malware.

Worse, not all resource compromises end up looking like malware. In the 2012
compromise of SourceForge's CDN, where a malicious version of phpMyAdmin was
delivered to hundreds of users, the PHP source itself was altered to enable
remote code execution. However, the extraneous code was virtually
indistinguishable from the rest of the raw PHP source in the phpmyAdmin image
being distributed.

At the time of writing (2018), VirusTotal correctly identifies the compromised
version of the Handbrake image as malware.

\subsubsection{HTTPS / Encrypted Channel}

The state of the art in secure communications is the Transport Layer Security
(TLS) standard and its Hyper Text Transfer Protocol (HTTP)/PKI based
implementation, HTTPS~\cite{TLS1.2, HTTPS, PKI}. Assuming well behaved
certificate authorities, modern browsing software, and proper deployment, TLS
and HTTPS mitigate many authentication and confidentiality related concerns
through the establishment of an encrypted authenticated communication channel
between parties~\cite{HTTPS, TLS1.2, DTLS}.

HTTPS and related protocols are not a panacea, however. Unlike concerns relating
to authenticity and confidentiality, \textit{resource integrity} deals with the
content of a communication; specifically: ensuring the bytes received at the end
of a transaction are the bytes we expected to receive. For example, a binary
expected to be 10MiB, when downloaded over the internet, should not be received
as an 11MiB executable---even if the compromised resource remained confidential
in transit between authenticated parties. Similarly, it would be ill advised to
knowingly execute a 10MiB binary that, for one reason or another, had half its
bits flipped by the time it was received. This can occur despite the integrity
guarantee provided by TLS/HTTPS because resource integrity attacks and other
SCAs fall outside of the threat model addressed by TLS/HTTPS.

Hence, despite the resilient nature of HTTPS and the powerful security
properties it guarantees, end users are still vulnerable to attacks on resource
integrity. This can occur at the point of distribution, such as a compromised
node in a third party CDN that delivers malicious resources masquerading as
popular software (see: cases 1-4 above). It can occur when an adversary gains
control of some aspect of the software deployment process (see: case 3 above).

Moreover, an adversary can render HTTPS-protected HTML-based protections such as
co-located resource checksums, the content-MD5 header~\cite{MD5Header} and
Subresource Integrity (SRI)~\cite{SRI} digests ineffective by trivially mutating
the compromised download portal and/or update/maintenance mechanism.

\subsubsection{Browser-based Heuristics and Blacklists}

Modern browsers employ heuristic and blacklist based detection and prevention
schemes to protect users from falling prey to malicious content on the internet.
Implementations include Google Chrome's \textit{Safe Browsing} feature, Mozilla
Firefox's \textit{Phishing/Download Protection}, and Microsoft Edge's
anti-malware \textit{sniffing} Windows Defender bundle.

Similar to anti-malware software, browser-based heuristics and blacklists are
reactive rather than proactive and, hence, ineffective at shielding users from
attacks on the integrity of the resources downloaded over the internet.

\PUNT{\subsubsection{Checksums}

\TODO{(what it is, why it fails; perhaps move part of abstract definition
here?)}}

\subsubsection{Public Key Infrastructure and Code Signing}

\TODO{(what it is, why it fails)}~\cite{DANE1, DANE2, DANE3, OpenPGP1}

\TODO{(specific case rather than the generic one; I think we should address the
differences explicitly: we're simpler; we do not make the system more fragile;
no central authority; applicable to more than just software binaries)}


