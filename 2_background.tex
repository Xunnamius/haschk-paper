\section{Background} \label{sec:background}

In this section we ...

\TODO{Use the language of IETF RFC3552 to describe active attack}

\subsection{Current Detection and Prevention Methods}

There are several. Blah blah.

\textbf{Anti-Malware Software.}    (what it is, why it fails; also talk about
manual scanning of files for viruses)

\textbf{HTTPS / Encrypted Channel.}    \cite{HTTP, HTTPS, TLS1, TLS2, DTLS}

\textbf{Browser-based Heuristics and Blacklists.}    (what it is, why it fails)

\textbf{Checksums.}    (what it is, why it fails)

\textbf{Public Key Infrastructure.}    \cite{DANE1, DANE2, DANE3, OpenPGP1}

\subsection{Motivation: Case Studies.}

Blurb about case studies.

\textbf{Case 1: x.}    Explain

\textbf{Case 2: x.}    Explain

\textbf{Case 3: x.}    Explain

\textbf{Case 4: x.}    Explain

\subsection{"Free" Highly-Available Distributed Systems.}

There might have been DNS size concerns, but EDNS/EDNS0 \cite{EDNS} assuages those fears.

Others are considering this as well, such as securitytxt draft \cite{draft-sectxt}. A widely
deployed example is DKIM \cite{DKIM}.
