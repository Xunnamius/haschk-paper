\section{Background} \label{sec:background}

\TODO{Use the language of IETF RFC3552 to describe active attack}

\subsection{Current Detection and Prevention Methods}

There are several. Blah blah.

\textbf{Anti-Malware Software}

(what it is, why it fails; also talk about manual scanning of files for
viruses)

\textbf{HTTPS / Encrypted Channel}

Fielding, R., Gettys, J., Mogul, J., Frystyk, H., Masinter, L., Leach, P. and T.
Berners-Lee, "HyperText Transfer Protocol", RFC 2616, June 1999.

Dierks, T. and C. Allen, "The TLS Protocol Version 1.0", RFC 2246, January 1999.

Rescorla, E., "HTTP over TLS", RFC 2818, May 2000.

Blake-Wilson, S., Nystrom, M., Hopwood, D. and J. Mikkelsen, "Transport Layer
Security (TLS) Extensions", RFC 3546, May 2003.

\textbf{Browser-based Heuristics and Blacklists}

(what it is, why it fails)

\textbf{Checksums}

(what it is, why it fails)

\textbf{Public Key Infrastructure}

Including DANE and similar technologies.

\subsection{Motivation: Case Studies}

(there are so many; enumerate them)

\subsection{"Free" Highly-Available Distributed Systems}

There might have been DNS size concerns, but EDNS/EDNS0 assuages those fears.

Others are considering this as well, such as securitytxt draft (cite). A widely
deployed example is DKIM. (cite).

tools.ietf.org/html/rfc6891
