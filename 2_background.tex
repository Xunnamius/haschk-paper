\section{Background} \label{sec:background}

In this section we ...

\TODO{Use the language of IETF RFC3552 to describe active attack}

\subsection{Current Detection and Prevention Solutions}

There are several. Blah blah.

\subsubsection{Anti-Malware Software.}    (what it is, why it fails; also talk
about manual scanning of files for viruses)

\subsubsection{HTTPS / Encrypted Channel.}   ~\cite{HTTP, HTTPS, TLS1, TLS2, DTLS}

\subsubsection{Browser-based Heuristics and Blacklists.}    (what it is, why it
fails)

\subsubsection{Checksums.}    (what it is, why it fails; perhaps move part of
abstract definition here?)

\subsubsection{Public Key Infrastructure.}   ~\cite{DANE1, DANE2, DANE3, OpenPGP1}

\subsection{Motivation: Case Studies.}

Blurb about case studies. \\

\noindent\textbf{Case 1: Floxif.} Explain \\

\noindent\textbf{Case 2: Kingslayer.} Explain \\

\noindent\textbf{Case 3: NotPetya.} Explain \\

\noindent\textbf{Case 4: Havex.} Explain \\

\subsection{``Free" Highly-Available Distributed Systems.}

Others are considering this as well, such as securitytxt
draft~\cite{draft-sectxt}. A widely deployed example is DKIM~\cite{DKIM}.
