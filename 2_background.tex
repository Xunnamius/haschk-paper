\section{Background} \label{sec:background}

\TODO{In this section we ...}

\TODO{Use the language of IETF RFC3552 to describe active attack}

\subsection{Motivation: Case Studies}

\TODO{Blurb about case studies. Information on breaches is spotty and incomplete
because corps are scant on the details of their failure, so we are working with
limited data.} \\

\noindent\textbf{Case 1: Linux Mint.} \TODO{Explain} \\

\noindent\textbf{Case 2: Havex.} \TODO{Explain} \\

\noindent\textbf{Case 3: PhpMyAdmin.} \TODO{Explain} \\

\noindent\textbf{Case 4: HandBrake.} \TODO{Explain} \\

\subsection{Current Detection and Prevention Solutions}

\TODO{There are several. Blah blah.}

\subsubsection{Anti-Malware Software}

\TODO{(what it is, why it fails; also talk about manual scanning of files for
viruses)}

\subsubsection{HTTPS / Encrypted Channel}

\TODO{(what it is, why it fails)}~\cite{HTTP, HTTPS, TLS1, TLS1.2, TLS0, DTLS}

\subsubsection{Browser-based Heuristics and Blacklists}

\TODO{(what it is, why it fails)}

\subsubsection{Checksums}

\TODO{(what it is, why it fails; perhaps move part of abstract definition
here?)}

\subsubsection{Public Key Infrastructure and Code Signing}

\TODO{(what it is, why it fails)}~\cite{DANE1, DANE2, DANE3, OpenPGP1}

\TODO{(specific case rather than the generic
one; I think we should address the differences explicitly: we're simpler; we
don't make the system more fragile; no central authority; applicable to more
than just software binaries)}


