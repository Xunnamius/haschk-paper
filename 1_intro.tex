\section{Introduction} \label{sec:introduction}

Downloading resources over the internet is a remarkably simple and painless
process for application developers and end users alike. The user (via their
browser) requests a server resource at some URL. The server responds with the
resource. The browser completes downloading the resource. Unfortunately,
downloading content over the internet can be risky.

\TODO{(the rest; an expository summarization of research)}

Discuss ``free," \ie no interface changes, no addition to resource download
time, no additional burden on the end user (qualified statement).

~\cite{DNSSEC}

In summary, our primary contributions are:

\begin{itemize}

  \item We propose a novel practical defense against receiving malicious,
  corrupted, or compromised resources over the internet. Contrasted with current
  solutions, our defense requires no UI or API interface changes at any level,
  does not employ unreliable heuristics, does not interfere with other software
  or extensions that also handle resource downloads, and can be transparently
  deployed without adding to the \textit{fragility} of the system. It protects
  end users whose software implements \SYSTEM{} while remaining unnoticeable to
  users of whose software does not. Hence, \SYSTEM{} could be deployed
  immediately.

  \item We present our prototype \SYSTEM{} implementations for Google Chrome and
  FileZilla and demonstrate its effectiveness in mitigating the consumption of
  compromised resources, even when transmitted over a secure channel. To the
  best of our knowledge, this is the \emph{first} system providing such
  capabilities with little implementation cost and at no cost to the end user.
  Further, we release the \SYSTEM{} solution to the community as open source
  software \footnote{The Chrome extension is available at
  https://tinyurl.com/dnschk-actual}.

  \item We carefully and extensively evaluate the security, scalability, and
  performance of our automated defense against resource corruption to
  demonstrate the effectiveness and high practicality of the \SYSTEM{} approach.
  Specifically, we find no obstacles to efficient scalability given choice of
  distributed system and no performance overhead compared to downloads without
  \SYSTEM{}. We further provide a publicly accessible demonstration of
  \SYSTEM{}'s utility via a patched HotCRP instance \footnote{The patched HotCRP
  instance is available at https://tinyurl.com/dnschk-hotcrp}.

\end{itemize}
