\section{Introduction} \label{sec:introduction}

Downloading resources over the internet is a remarkably simple and painless
process for application developers and end users alike. The user (via their
browser) requests a server resource at some URL. The server responds with the
resource. The browser completes downloading the resource. Unfortunately,
downloading content over the internet can be risky.

\TODO{(the rest; an expository summarization of research)}

Discuss "free": no interface changes, no addition to resource download time, no
additional burden on the end user (qualified statement).

DNSSEC: Eastlake, D., "Domain Name System Security Extensions", RFC 2535, March
1999.

In summary, our primary contributions are:

\begin{itemize}

  \item We propose to \TODO{the DNSCHK approach using DNS/DHT} ...

  \item We present our prototype DNSCHK implementations for Google Chrome and
  FileZilla. To the best of our knowledge, this is the \emph{first} system
  having such capabilities. Further, we release the DNSCHK solution to the
  community as open source software (footnote with link).

  \item We carefully and extensively evaluate the security, scalability, and
  performance of our automated defense against resource corruption to
  demonstrate the effectiveness and high practicality of the DNSCHK approach. We
  further provide a proof-of-concept publicly accessible demonstration of
  DNSCHK's utility via a patched HotCRP instance (footnote with link).

\end{itemize}
