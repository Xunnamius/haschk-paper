\section{Introduction} \label{sec:introduction}

Downloading resources over the internet is a remarkably simple and painless
process for application developers and end users alike. The user (via their
browser) requests a server resource at some URL. The server responds with the
resource. The browser completes downloading the resource. Unfortunately,
downloading content over the internet can be risky.

Supply Chain Attacks (SCA) are the compromise of software source code via cyber
attack, insider threat, or other attack on one or more phases of the development
and deployment life cycle. These attacks are made possible due to proximity and
have the goal of infecting and exploiting one or more victims---usually the
software company's customer base. \SYSTEM{} only protects against SCAs that
occur after the Authoritative Hash (AH) is calculated. AH calculation is
necessarily more likely to occur later in the software development life cycle or
very early in the deployment process. If an attacker is able to execute a
successful SCA before the AH is calculated, \SYSTEM{} would propagate the
compromised Authoritative Hash.

\TODO{(include table of supply chain stages that \SYSTEM{} is effective in)}

Although early Supply Chain Attacks are devastating, they are not the only
popular form of the attack. Many devastating supply chain attacks occur late in
the software development and deployment process, including \TODO{(choose three examples)}. Further, \TODO{(popular attacks at various levels, including CDNs)}.

Discuss ``free," \ie no interface changes, no addition to resource download
time, no additional burden on the end user (qualified statement).~\cite{DNSSEC}

In summary, our primary contributions are:

\begin{itemize}

  \item We propose a novel practical defense against receiving malicious,
  corrupted, or compromised resources over the internet. Contrasted with current
  solutions, our defense requires no source code or infrastructure changes at
  any level other than DNS, does not employ unreliable heuristics, does not
  interfere with other software or extensions that also handle resource
  downloads, and can be transparently deployed without adding to the
  \textit{fragility} of DNSSEC-enabled systems; it protects end users whose
  software implements \SYSTEM{} while remaining unnoticeable to users of whose
  software does not.

  \item We present our prototype \SYSTEM{} implementations for Google Chrome and
  FileZilla and demonstrate its effectiveness in automatically and transparently
  mitigating the accidental consumption of compromised resources from a
  compromised server hosting a compromised web portal. To the best of our
  knowledge, this is the \emph{first} system providing such capabilities with
  little implementation cost and at no cost to the end user. Further, we release
  the \SYSTEM{} solution to the community as open source software\footnote{The
  Chrome extension is available at https://tinyurl.com/dnschk-actual}.

  \item We carefully and extensively evaluate the security, scalability, and
  performance of our automated defense against resource corruption to
  demonstrate the effectiveness and high practicality of the \SYSTEM{} approach.
  Specifically, we find no obstacles to efficient scalability given choice of
  distributed system and no performance overhead compared to downloads without
  \SYSTEM{}. We further provide a publicly accessible demonstration of
  \SYSTEM{}'s utility via a patched HotCRP instance\footnote{The patched HotCRP
  instance is available at https://tinyurl.com/dnschk-hotcrp}.

\end{itemize}
