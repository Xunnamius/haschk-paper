\section{Introduction} \label{sec:introduction}

\begin{table*}[th]
  \centering
  \begin{tabular}{|*{10}{c|}}
    \hline\textbf{Concept}
        & \textbf{Design} & \textbf{Development} & \textbf{Integration} &
        \textbf{Deployment} & \textbf{Maintenance} & \textbf{Retirement}\\\hline
    X&X&X&X&\ding{51}&\ding{51}&\ding{51}\\\hline
  \end{tabular}
  \caption{Supply Chain Attack mitigation opportunities for \SYSTEM{} in the
  software development and deployment life cycle}\label{tbl:attacks}
\end{table*}

\TODO{I have some notes scattered through the intro, but I think this
  whole section needs a bit of reorganization to tighten it up. I think you need the following structure:
\\ \\
Para1: Downloading is easy but risky and the risk is dividided into three categories. \\ \\
Para2: Narrow scope of the paper to third risk, and relate to supply chain attacks.  \\ \\
Para3: State of the art dealing with supply chain attacks: checksums. \\ \\
Para4: Why checksums are bad (may be folded into previous para): (1) tedious for users and (2) easily compromised by many existing supply chain attacks (not sure about the last one). \\ \\
Para5: DNSCHK as replacement for checksums that addresses both problems through (1) simple extensions to browsers or other tools and (2) whatever the general name is for the backend stuff which can be implemented on DNSSEC or a DHT (or maybe even something else we haven't thought of). \\ \\
Para6: How do we evaluate it. \\ \\ 
Para7: List of contributions at end of current intro. }


Using the internet to receive data is a remarkably simple and painless process
for application developers and end users alike. When using a browser such as
Google Chrome, this standard and familiar process can be summarized as: 1) a
user requests a server resource at some URL, 2) the server responds with the
desired resource, and finally 3) the browser completes downloading the resource.

Unfortunately, and as most are already aware, downloading content over
the internet can be extremely risky no matter the protocol.
\TODO{Following sentence needs one or more citations to back up the
  claim that this is generally how things are done}. We can generally
divide this risk into three categories: response authentication,
communication confidentiality, and resource integrity. Response
authentication allows us to determine (usually asymmetrically) if a
response received at some destination indeed originates from its
purported source. This is typically accomplished through the adoption
of some Public Key Infrastructure (PKI) scheme.  Communication
confidentiality, on the other hand, allows us to keep the data
transacted between two or more parties private except to said parties.
This is typically accomplished through some combination of symmetric
and asymmetric encryption. Finally, resource integrity allows us to
verify that the data we are receiving is the data we are expecting to
receive.

When it comes to response authentication and communication confidentiality, the
state of the art in production attack mitigation is Transport Layer Security
(TLS) and its Hyper Text Transfer Protocol (HTTP)/PKI based implementation,
HTTPS~\cite{TLS1.2, TLS1, TLS0}. Assuming well behaved certificate authorities
and modern browsing software, a web server with HTTPS properly deployed
mitigates myriad attack classes ranging from DNS-related and Man-in-the-Middle
attacks to data theft, communication forgery, and sender repudiation.

\TODO{I know this is pedantic as hell, but you don't want to start two
  paragraphs in the intro with ``Unfortunately,'' because that word is
  a signal that you are about to introduce the problem the paper
  solves.  I think the issue is easily fixed in this instance by
  changing the first sentence of the second paragraph to just start
  with ``Downloading content over the internet can be extremely
  risky...'' }  Unfortunately, HTTPS and related protocols (HSTS,
DNSSEC, etc) are not a panacea. Unlike authenticity and
confidentiality, \textit{resource integrity} deals with the content of
a communication; specifically: ensuring the bytes received at the end
of a transaction are the bytes we expected to receive. For example, a
binary expected to be 10MiB, when downloaded over the internet, should
not be received as an 11MiB executable, even if the communication was
confidential between parties and the receiver can verify that the
resource came from the intended source. Similarly, it would be ill
advised to execute a 10MiB binary that, for one reason or another, had
half its bits flipped by the time it was received. This can occur
despite the integrity guarantee provided by TLS/HTTPS because, as a
\textit{communication} protocol, TLS only guarantees the integrity of
each \textit{end to end communication} as an opaque payload via
message authentication code (MAC). The integrity of the original
payload as it pertains to the Application layer rather than the
Transport layer is outside of the model addressed by TLS and
HTTPS~\cite{TLS1.2, HTTPS}.  \TODO{I actually think you still don't
  want the ``Unfortunately'' on this paragraph either.  I believe you
  want the problem to be the current use of checksums to solve the
  resource integrity problem, so the next paragraph should be the one
  that gets the ``Unfortunately...''}

Hence, despite the resilient nature of HTTPS and the powerful security
properties it guarantees, end users are still vulnerable to attacks on resource
integrity. This can occur at the point of distribution, such as a compromised
node in a third party CDN that delivers compromised malicious resources
masquerading as popular public software libraries. \TODO{(choose an example from
background section)}. It can occur when an adversary gains control of some
aspect of the software deployment process, such as \TODO{(talk about
NotPetya?)}. Moreover, an adversary can render HTML-based additions to HTTPS
such as Content-MD5 header~\cite{MD5Header} and Subresource Integrity
(SRI)~\cite{SRI} ineffective by compromising the system that hosts the software
or its download portal or update/maintenance mechanism. Protected encrypted
communications mean nothing if the contents of those communications are
themselves corrupted before the fact.

More broadly, these kinds of resource integrity attacks are known as
\textbf{Supply Chain Attacks} (SCA)~\cite{}. SCAs are the compromise of software
source code via cyber attack, insider threat, or other attack on one or more
phases of the software development life cycle (see \figref{attacks}). These
attacks have the goal of infecting and exploiting one or more target victims.

Ensuring the integrity of resources exchanged over the internet despite SCAs and
other active attacks is an incredibly well studied problem~\cite{MD5Header,
HTTP1.1, HTTPS, SRI, LF, OpenPGP1, DNSSEC}. The state of the art is checksums.
\TODO{(expoud on checksums; link it to the attacks figure)}.

However, checksums are not a viable solution. \TODO{(expound on ineffectiveness;
use Linux Mint example here?)}.

In this paper, we propose \SYSTEM{}, \TODO{(deliniate between approach and
implementations?)}. Unlike \TODO{(compare and contrast)}.

As a replacement for manual checksum verification, \SYSTEM{} addresses a similar
security model and   (see \tblref{attacks}). AH calculation is necessarily more
likely to occur later in the software development life cycle or very early in
the deployment process. If an attacker is able to execute a successful SCA
before the AH is calculated, \SYSTEM{} would propagate the compromised
Authoritative Hash.

\TODO{Discuss ``free," \ie no interface changes, no addition to resource
download time, no additional burden on the end user (qualified
statement)}~\cite{DNSSEC}.

\TODO{Although the process sounds simple and intuitive, X practical questions
must be addressed (grab from and potentially shrink the new abstract?)}

In summary, our primary contributions are:

\begin{itemize}

  \item We propose a novel practical defense against receiving malicious,
  corrupted, or compromised resources over the internet. Contrasted with current
  solutions, our defense requires no source code or infrastructure changes at
  any level other than DNS, does not employ unreliable heuristics, does not
  interfere with other software or extensions that also handle resource
  downloads, and can be transparently deployed without adding to the
  \textit{fragility} of DNSSEC-enabled systems; it protects end users whose
  software implements \SYSTEM{} while remaining unnoticeable to users of whose
  software does not.

  \item We present our prototype \SYSTEM{} implementations for Google Chrome and
  FileZilla and demonstrate its effectiveness in automatically and transparently
  mitigating the accidental consumption of compromised resources from a
  compromised server hosting a compromised web portal. To the best of our
  knowledge, this is the \emph{first} system providing such capabilities with
  little implementation cost and at no cost to the end user.

  \item We carefully and extensively evaluate the security, scalability, and
  performance of our automated defense against resource corruption to
  demonstrate the effectiveness and high practicality of the \SYSTEM{} approach.
  Specifically, we find no obstacles to efficient scalability given choice of
  distributed system and no performance overhead compared to downloads without
  \SYSTEM{}. We further provide a publicly accessible empirical demonstration of
  \SYSTEM{}'s protective utility via a patched HotCRP instance\footnote{The
  patched HotCRP instance is available at https://tinyurl.com/dnschk-hotcrp}.

\end{itemize}

We release the \SYSTEM{} solution to the community as open source software to
prompt exploration of the \SYSTEM{} approach\footnote{The \SYSTEM{} Chrome
extension is available at https://tinyurl.com/dnschk-actual}.
