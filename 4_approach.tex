\section{The \SYSTEM{} Approach} \label{sec:approach}

In this section we detail the \SYSTEM{} approach, a practical defense against
receiving corrupted or compromised resources over the internet that 1) is
automated to account for user apathy and 2) does not require co-hosting the
checksum, which creates a single point of failure and renders checksum
verification (even automated checksum verification) irrelevant.

To achieve these goals, we first require some method to uniquely identify
individual resources. This can be addressed with checksums themselves, as they
are the output of cryptographic hashing functions that are pre-image, second
pre-image, and collision resistant. So long as the chosen function is not known
to be weak, the output can be assumed unique per resource~\cite{Rogaway}. Given
this, we adopt the informal IETF draft for hash-based Uniform Resource Name
(URN) namespaces~\cite{draft-URN} to both uniquely identify resources and, when
combined with an \emph{origin domain} (below), additionally associate the
resource with the entity distributing it. URNs make this association durable: an
entity uses a high availability mapping to publicly advertise the URNs of
resources they offer for download, which can identify resources whose access
path or URI is not stable (\eg{resources hosted on download mirrors and CDNs}).

Since the high availability system advertising URNs is not co-hosted with the
system that distributes the corresponding resources---like a web or FTP
server---\SYSTEM{} retains the ability to protect users from dangerous downloads
\emph{even when the system distributing the download has been completely
compromised}. This is not true of prior approaches to automated checksum
verification of arbitrary resources.

\subsection{Overview}

\SYSTEM{} implementations consist of two parts: the actual client---like Google
Chrome or Filezilla---and the high availability system against which URN queries
are made. Immediately after a resource is obtained by the client---like when a
download link is clicked---a ``non-authoritative'' cryptographic digest
(checksum) is generated by running a cryptographic hashing function over the
contents of the resource file. This checksum is used to construct the URN that
uniquely fingerprints that resource in accordance with the draft
spec~\cite{draft-URN}. The URN can be further encoded as individual
implementations demand. For instance, our \DNSSYS{} implementation replaces all
colons (``:'') in the URN with dashes and removes all padding characters so that
URNs form valid URLs.

Next, \SYSTEM{} uses the URN and the origin domain to query the high
availability system. At this point, the system will respond with either 1) a
confirmation that the URN exists, in which case \SYSTEM{} determines the
download to be safe or 2) a rejection when the URN is not found, in which case
\SYSTEM{} determines the download to be unsafe. In the case where an ``unsafe''
determination is made, some implementation-specific action should be taken that
accounts for user apathy to mitigate risk.

\subsection{Accounting for User Apathy}

Human factors such as user apathy have stymied cryptographers for decades.
Schemes that are otherwise reasonably cryptographically solid can fail
catastrophically due to human error, confusion, aversion to inconvenience, or
simple lack of interest. Some users are likely to avoid using a security measure
altogether if it presents even a minor obstacle to immediate
gratification~\cite{Clickthrough, PGPBad, Egelman1, Egelman2, Jenkins, Modic,
Reeder, Silic, Sunshine, Bianchi, Akhawe, Cherubini}.

With this in mind, the primary goal of \SYSTEM{} is to side-step the human
factor by providing a completely transparent and unobtrusive, fully-automated
method of checksum calculation and verification in the common case. In the
uncommon case, when a download is determined ``unsafe'', we 1) clearly, visibly,
and simply assert the danger of the download and 2) value user security over
choice as suggested by Cherubini et al~\cite{Cherubini}. This means deleting,
renaming, or otherwise making the unsafe resource inaccessible by default and
forcing the user to confront the problem with no easy avenue to click-through
the warning. In the more common case where a ``safe'' determination is made,
\SYSTEM{} should remain unobtrusive to the user.

\subsection{Mitigating the Co-Hosting Problem}

Funding and maintaining a single server/system to host assets can be extremely
cost-effective in the short term compared to hosting two or more discrete
systems, such as one to host a resource and another to host a resource's
checksum. Unfortunately, this creates a single point of failure: an attacker
that compromises this system can both corrupt the resource and update the
checksum to match. Hence, co-hosting a resource and its corresponding checksum
on the same system virtually negates the effectiveness of having a checksum at
all.

Hence, proper deployment of the \SYSTEM{} approach necessitates the existence of
a separate distribution mechanism for advertising ``safe'' URNs. The idea of
using some distributed storage service to query a global mapping of such values
is not new and may seem straightforward, but there are some deceptively complex
implementation challenges.

Foremost is determining which high availability system should be queried for a
given resource. We refer \emph{origin domain} for a download. \TODO{Explain.
Also add: see implementations.}

An additional challenge is the choice of high availability system to host an
entity's URN mappings. There has been a lot of effort put into researching,
designing, and standardizing several high availability high performance storage
technologies, some of which web-facing entities and IT teams are already quite
familiar with and pay for, \eg the Domain Name System (DNS). Other candidate
high availability systems include DHTs, storage clusters, relational and
non-relational databases, and any high availability key-value store reasonably
capable of guaranteeing the authenticity of its responses.
