\section{The \SYSTEM{} Approach} \label{sec:approach}

In this section we detail the \SYSTEM{} approach: a novel defense against
receiving corrupted or compromised resources over the internet. We further
present the challenges and their solutions in designing \SYSTEM{} that
transparently mitigates resource integrity Supply Chain Attacks (SCA) without
degrading the experience of users that do not implement the \SYSTEM{} approach.

Further, though our concrete implementation relies on DNS and DNS Security
(DNSSEC), the approach itself is flexible and completely agnostic of any single
component. The implementation choice of highly-available distributed backend is
not restricted to the DNS network. The approach works just as well with, for
instance, an authenticated Distributed Hash Tables (DHT) or some distributed
authenticated key-value store (\eg Redis).

\TODO{talk about the Chrome/FileZilla extensions as addressing user apathy with
the DNSSEC/DHT implementations addressing co-hosting.}

\TODO{Resource Identifier (RI)}  
\TODO{Authoritative Hash (AH)}  
\TODO{Non-Authoritative Hash (NAH)}  
\TODO{Non-Authoritative Hash Validation (NAH Validation)}  
\TODO{Origin Domain (OD)}  
\TODO{Primary Label}  
\TODO{RI Sub-Label}

\subsection{Transparency and User Apathy}



\subsection{Defeating Co-Hosting for ``Free''}



\subsection{Platform Diversity}



\subsection{Proof-of-Concept Implementations}

\TODO{(Google Chrome Extension: describe implementation details; works with DNS
or DHT and is published to Chrome store; no interface changes!--i.e. downloads
work exactly the same with or without the extension; users still have to
confirm/deny suspicious judgements, but they are rare occurrences)}

\TODO{Reference hotcrp demo but leave the description for the evaluation.}

\TODO{(FileZilla (FTP) Patch (only talked about here) describe implementation
details; minor interface change if the download is judged unsafe or suspicious,
requires user to confirm/deny download)}
