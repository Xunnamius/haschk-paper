\section{The \SYSTEM{} Approach} \label{sec:approach}

In this section we detail the \SYSTEM{} approach: a novel defense against
receiving corrupted or compromised resources over the internet. We further
present the challenges and their solutions in designing \SYSTEM{} that
transparently mitigates resource integrity Supply Chain Attacks (SCA) without
degrading the experience of users that do not implement the \SYSTEM{} approach.

Further, though our concrete implementations relies on DNS authenticated with
DNS Security (DNSSEC), the approach itself is flexible and completely agnostic
of any single component. The implementation choice of highly-available
distributed \emph{backend}, for instance, is not restricted to the DNS network.
The approach works just as well with an authenticated Distributed Hash Table
(DHT) or some distributed authenticated key-value store (\eg Redis) as the
backend.

\TODO{Create the overview image or remove this sentence} \figref{overview}
illustrates the \SYSTEM{} approach.

\TODO{talk about the Chrome/FileZilla extensions as addressing user apathy with
the DNSSEC/DHT implementations addressing co-hosting.}

\PUNT{Frontend:}
\subsection{Transparency and User Apathy}

for most users, hand-calculating the checksum and comparing to the published
version is too off-path and tedious.

PGP apathy example PGP apathy applies here It is a mistake to pit security
concerns against human factors like this.

The primary goal of \SYSTEM{} is to side-step the problem altogether. Full
automation and transparency for the end user.

We achieve this by

A typical flow can be summarized as (now we introduce some jargon)

Maybe a figure too as well?

\PUNT{Backend:}
\subsection{Defeating Co-Hosting, Sometimes for Free}

Cohosting is bad. Reiterate why (from intro).

Though the concept of using some distributed authenticated storage mechanism to
query a global mapping between RIs and AHs sounds intuitive and straightforward,
two natural concerns arise.

The first: modern durable authentication schemes are based on PKI; who is
managing the keypairs?

The second: who is paying to maintain this secondary system? Who is bearing the
burden of its maintenance?

\subsection{Platform Diversity}

Not computationally intensive. Requests are small, low network load. See eval.

Hence, the approach can be incorporated into software on any device capable of
communicating with a distributed authenticated backend. This includes desktops,
laptops, tablets, mobile devices, embedded systems.

\subsection{Proof-of-Concept Implementations}

A Resource Identifier (RI) is the unique cryptographic digest yielded by hashing
the full file path of the resource, including leading slash if applicable. For
example, considering a web resource hosted at
\textit{https://somesite.com/downloadme.txt}, a browser-based \SYSTEM{}
implementation would hash \textit{/downloadme.txt} to get the RI.

The Authoritative Hash (AH) is yielded from a record request to the backend.

The Non-Authoritative Hash (NAH) is calculated by hashing the contents of the
resource after it has been received in its entirety.

Non-Authoritative Hash Validation (NAH Validation) is the final stage before
\SYSTEM{} renders judgement on the downloaded resource's integrity. The NAH is
compared to the AH received from backend. If they do not match, the file is
judged unsafe.

The Origin Domain (OD) is used to determine how \SYSTEM{} communicates with the
backend. In a browser-based implementation, the OD is the Second-Level Domain
(SLD) fragment of the current tab's URL. For example:
\emph{google.com} would be the OD for the URL \emph{frag.something.google.com}
and \emph{fakesite.io} would be the OD for the URL \emph{git.fakesite.io}.

\subsubsection{Google Chrome}

\TODO{(Google Chrome Extension: describe implementation details; works with DNS
or DHT and is published to Chrome store; no interface changes!--i.e. downloads
work exactly the same with or without the extension; users still have to
confirm/deny suspicious judgements, but they are rare occurrences)}

The Primary Label (PL) is a standard string used to identify DNS records that
belong to DNSCHK. It will always appear as the third-level domain following the
OD.

The RI Sub-Label (SL) is a standard string used to identify DNS records that
contain RIs.

Chrome: the origin domain is determined via DownloadItem::referrer. This is to
make it harder for an adversary to trick DNSCHK into calculating an incorrect
OD.

\TODO{Reference hotcrp demo but leave the description for the evaluation.}

\begin{algorithm}[t]
    %\floatname{algorithm}{Algorithm}
    \caption{Handling an incoming download in Chrome} \label{algo:dnschk}
    {\footnotesize
    \begin{algorithmic}[1]
    \Require The read request is over a contiguous segment of the backing
    store
    \Require $\ell, \ell' \leftarrow$ read requested length
    \Require $\aleph \leftarrow$ master secret
    \Require $n_{index} \leftarrow$ first nugget index to be read
    \State $data \leftarrow$ \emph{empty}
    \While{$\ell \neq 0$}
        \State $k_{n_{index}} \leftarrow GenKey_{nugget}(n_{index}, \aleph)$
        \State Fetch nugget keycount $n_{kc}$ from Keycount Store.
        \State Calculate indices touched by request: $f_{first}$, $f_{last}$
        \State $n_{flakedat} \leftarrow ReadFlakes(f_{first},\dots,f_{last})$
        \For{$f_{current} = f_{first}$ \textbf{to} $f_{last}$}
            \State $k_{f_{current}} \leftarrow GenKey_{flake}(k_{n_{index}},
            f_{current}, n_{kc})$
            \State $tag_{f_{current}} \leftarrow GenMac(k_{f_{current}},
            n_{flakedat}[f_{current}])$
            \State Verify $tag_{f_{current}}$ in Merkle Tree.
        \EndFor
        \LineComment{(\textbf{*}) denotes requested subset of nugget data}
        \State $data \leftarrow data + Decrypt(*n_{flakedat}, k_{n_{index}},
        n_{kc})$
        \State $\ell \leftarrow \ell - \|*n_{flakedat}\|$
        \State $n_{index} \leftarrow n_{index} + 1$
    \EndWhile
    \\\Return $data$ \Comment{Fulfill the read request}
    \Ensure $\|data\| <= \ell'$
    \Ensure $\ell = 0$
    \vskip -1.5em
    \end{algorithmic}
    }
\end{algorithm}

\subsubsection{Filezilla}

\TODO{(FileZilla (FTP) Patch (only talked about here) describe implementation
details; minor interface change if the download is judged unsafe or suspicious,
requires user to confirm/deny download)}
