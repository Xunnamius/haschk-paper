\section{The \SYSTEM{} Approach} \label{sec:approach}

In this section we detail the \SYSTEM{} approach: a practical defense against
receiving corrupted or compromised resources over the internet. We further
present our proof-of-concept Google Chrome extension implementations, \DNSSYS{}
and \DHTSYS{}.

\figref{overview} illustrates the \SYSTEM{} approach.

\subsection{Defeating User Apathy}

Human factors such as user apathy have stymied cryptographers for decades.
Schemes that are otherwise reasonably cryptographically solid can fail
catastrophically due to human error, confusion, or simple lack of interest. Some
users are likely to avoid using a security measure altogether if it presents
even a minor obstacle to immediate gratification~\cite{Clickthrough, PGPBad}. In
the browser, for example, this phenomenon can be observed empirically.

Leveraging the in-browser telemetry of Mozilla Firefox and Google Chrome to
passively observe over 25 million warning impressions in 2013 (see
\figref{telemetry}), Akhawe et al. found that users of Google Chrome clicked
through a \emph{quarter of malware and phishing warnings} and \emph{70\% of TLS
warnings}~\cite{Clickthrough}. Users also clicked through a third of Mozilla
Firefox's TLS warnings and a tenth of their malware and phishing warnings. That
is to say: a significant percentage of browser users are \emph{determined} not
to let TLS trust issues and/or the threat of malware prevent them from receiving
their desired content. Hence, we must assume: some non-trivial number of users,
similarly determined to transact resources over the internet, will not be
burdened with the off-path minutiae of manually calculating a checksum (if they
are even familiar with the jargon) and verifying the integrity of the resources
they are downloading.

With this assumption in mind, the primary goal of \SYSTEM{} then is to side-step
the human factor altogether by providing a completely transparent and
unobtrusive, fully-automated method of checksum calculation and verification in
the average case. We achieve this through 1) the unique identification of
individual hosted resources and 2) a highly available mapping of unique resource
identifiers to corresponding checksums.

\figref{overview} illustrates the \SYSTEM{} approach. \SYSTEM{} implementations
can be imagined as a security layer sitting between the user and the resource.
Immediately after a resource is downloaded, two cryptographic digests are
generated. One digest uniquely fingerprints said resource based on its name.
This is known as the \emph{Non-Authoritative Checksum} (NAC) and is yielded from
running the cryptographic hashing function over the contents of the resource
file. The second digest uniquely fingerprints said resource based on its
contents. This is known as the \emph{Resource Identifier} (RI) and is yielded
from running the cryptographic hashing function over the resource's public path
on the distribution system.

Next, \SYSTEM{} uses the RI to retrieve an \emph{Authoritative Checksum} (AC)
from the backend. If successful, \SYSTEM{} will compare the NAC to the AC---we
refer to this as \emph{Non-Authoritative Checksum Validation} (NAC Validation).
Only in the case where NAC Validation fails, \ie they do not match, will the
user realize \SYSTEM{} exists. Otherwise, \SYSTEM{} remains completely
transparent the the end user, as demonstrated by our browser-based
implementations.

Note that we consider URNs as an alternative fingerprinting scheme that combines
RIs and NACs in the discussion (cf. \secref{discussion}).

\subsection{Defeating Co-Hosting}

Funding and maintaining a single server/system to host all of your assets can be
extremely cost-effective in the short term compared to hosting two or more
discrete systems---one hosting the resource and one hosting the resource's
checksum. Unfortunately, this establishes a single point of failure: an attacker
that compromises such a system can both mutate the resource and update the
checksum to match the mutation. Hence, \emph{co-hosting} a resource and its
corresponding checksum on the same distribution system virtually negates the
effectiveness of having a checksum at all. This is widely understood in the
security community~\cite{SCA-MINT2}.

Hence, deployment of \SYSTEM{} necessitates the existence of a separate
distribution mechanism for resources and ACs. Though the concept of using some
distributed authenticated storage service to query a global mapping between RIs
and ACs sounds intuitive and straightforward, two natural concerns arise. The
first: modern fully authenticated schemes are based on PKI; who is managing this
infrastructure and can they be trusted? The second: who is funding the
establishment and maintenance of this potentially complex secondary system?

Fortunately, there exists a highly-available fully authenticated globally
distributed high performance low latency mapping service that web-facing
entities and IT teams are already quite familiar with (and already pay for): the
Domain Name System (DNS). Adding extra resource records to a DNS zone is
essentially a costless operation, meaning any entity that already has a
DNSSEC-protected web presence can immediately deploy \SYSTEM{}.

\subsection{Platform Diversity}

From our evaluation, the computational overhead of running \SYSTEM{} is minimal
for most resources. Further, additional network load is negligible (cf.
\secref{evaluation}). Hence, the \SYSTEM{} approach can be incorporated into
software on most any device capable of communicating with the chosen backend.
This includes desktops, laptops, tablets, mobile devices, embedded systems, etc.

\subsection{Proof-of-Concept Implementations}

\subsubsection{\DNSSYS{} and \DHTSYS{}}

We implement \SYSTEM{} as two proof-of-concept Google Chrome extensions:
\DNSSYS{} and \DHTSYS{}. They work with DNS and Ring OpenDHT as their highly
available backends, respectively. Our Chrome extensions do not make any
modifications to the Chrome user interface or viewport other than the extension
icon in the toolbar. Further, downloads work exactly the same whether or not
\DNSSYS{} or \DHTSYS{} are installed; the extensions are transparent to end
users. However, if a failure is experienced during NAC Validation (\ie a
``non-average'' case), the extensions will alert the user to the dangerous
download via toolbar icon and popup interface.

Immediately after a resource download is first detected, the extensions compute
an RI from the full URL path of the resource. For the purposes of our
proof-of-concept implementations, we calculate the RI as a hash digest of the
path component of the URL pointing to the resource. For example, consider a web
resource hosted at \texttt{https://somesite.com/var/downloadme.txt}. Our
implementations would hash \texttt{/var/downloadme.txt} to yield an RI. Note
that there are several ways a browser extension could calculate an RI. See
\secref{discussion} for a discussion of a more robust RI calculation strategy.

Next, we fetch the so-called \emph{Origin Domain} (OD). The OD is the base
domain used to query the backend, and will always be the Second-Level Domain
(SLD) fragment of the active browser tab's URL in our implementations. For
example: \texttt{somesite.com} would be the OD for the URL
\texttt{frag.something.somesite.com} and \texttt{fakesite.io} would be the OD
for the URL \texttt{git.fakesite.io}.

The OD is then appended to the Primary Label (PL), which is then appended to the
RI Sub-Label (SL). The PL is a standard string used to more easily identify the
backend records our implementations rely upon; we used ``\_dnschk''. It will
always appear as the third-level domain following the OD in any request to the
backend. The SL is a standard string used to identify DNS records that contain
RIs; we used ``\_ri'' in our implementations. The resulting construction,
consisting of \texttt{SL.PL.OD} (\eg \texttt{\_ri.\_dnschk.fakesite.io}), is
appended to the RI calculated earlier. This forms the subject of the query to
our backend, whereafter the backend responds with the AC or an indication that
the RI-to-AC mapping was not found.

To remain in compliance with DNS protocol label limits, we chose to split the
RI---a 64 character string---into two labels separated by a period. We do this
for both implementations, though it is only relevant with \DNSSYS{}.

The final query sent to the backend consists of \texttt{RI1.RI2.SL.PL.OD}.
Continuing with our previous example, that yields: \\ \texttt{RIPart1.RIPart2.\_ri.\_dnschk.fakesite.io}.

If the backend responds with the AC and NAC Validation succeeds, our extensions
render a ``safe'' judgement via the extension UI. If the backend responds with
the AC but NAC Validation fails, our extensions render an ``unsafe'' judgement.
If the backend response indicates the RI-to-AC mapping was not found, there are
two possible outcomes: the extensions render a ``neutral'' judgement if they are
not operating under strict mode conditions, otherwise it will render an
``unsafe'' judgement (as if NAC Validation had taken place and failed).

``Strict mode'' status, if active, prevents \DNSSYS{} and \DHTSYS{} from
rendering ``neutral'' judgements for a particular OD. Normally, neutral
judgements allow our extensions to be deployed immediately on the open internet
without pestering the end user with false positives when downloading resources
that are not protected by our approach. For resources that are, strict mode
ensures the judgements rendered by \DNSSYS{} and \DHTSYS{} for a given OD remain
binary: ``safe'' if NAC Validation succeeds or ``unsafe'' in all other
circumstances.

To determine if strict mode status applies to an OD, an additional backend query
is made of the form \texttt{SML.PL.OD}, where SML is the Strict Mode Sub-Label
consisting of the standard string ``\_smode''. Continuing with our previous
example, our query would take the form \texttt{\_smode.\_dnschk.fakesite.io}. If
and only if the subject of the query does in fact exist in the backend, strict
mode status is assumed.

\subsubsection{Determining Origin Domain in the Browser}

If an attacker manages to compromise a web page/server, they have two options.
They can mutate the resource directly, which would be observable via \SYSTEM{}.
They could also mutate the web/download page itself, replacing the anchor with a
malicious one that points to a compromised resource on the attacker's remote
system. \SYSTEM{} will still catch this due to the Chrome API's
\texttt{DownloadItem::referrer} property (distinct from the concept of an HTTP
referrer).

However, a clever attacker might be able to trick the Chrome API into populating
\texttt{DownloadItem::referrer} by redirecting the user to a valid and innocuous
page that very quickly redirects the user again to a compromised resource with
the goal of tricking the Chrome API into supplying \SYSTEM{} with a chosen
\texttt{DownloadItem::referrer}.

In order to prevent such implementation-specific attacks, we make a distinction
between the domain that the hyperlink containing the desired resource references
and the Origin Domain (OD)---or the domain of the document within which said
hyperlink exists. The extension is implemented such that the OD is computed as
early as possible in the page loading process. The scope of the OD is at the tab
level, meaning there is one OD determined for each open browser tab. Once
determined for a tab, the OD should not be recalculated for some period of time.
If the tab is navigated and a download is started within a chosen time window,
the user will be asked to verify that the OD is what they expect it to be
(should be a familiar URL).

We catch potential redirection attacks by assuming any navigation that results
in a direct download up to three (3) seconds after the page has loaded is
suspicious and requires affirmation by the user. We expect this extreme
Chrome-specific edge case to occurrence very rarely, if at all.
