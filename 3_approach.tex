\section{The \SYSTEM{} Approach} \label{sec:approach}

\TODO{In this section we ...}

\TODO{Resource Identifier (RI)}
\TODO{Authoritative Hash (AH)}
\TODO{Non-Authoritative Hash (NAH)}
\TODO{Non-Authoritative Hash Validation (NAH Validation)}
\TODO{Origin Domain (OD)}
\TODO{Primary Label}
\TODO{RI Sub-Label}

\TODO{(describe generalized solution system using any sort of distributed
highly-available key-value store that exists or could exist (like dns))}

\TODO{Split explanation into "frontend" concerns and "backend" concerns}

\TODO{Go over algorithms!}

\TODO{talk about the Chrome/FileZilla extensions as addressing user apathy with
the DNSSEC/DHT implementations addressing co-hosting.}

\subsection{Proof-of-Concept Implementations}

\TODO{(Google Chrome Extension: describe implementation details; works with DNS
or DHT and is published to Chrome store; no interface changes!--i.e. downloads
work exactly the same with or without the extension; users still have to
confirm/deny suspicious judgements, but they are rare occurrences)}

\TODO{Reference hotcrp demo but leave the description for the evaluation.}

\TODO{(FileZilla (FTP) Patch (only talked about here) describe implementation details; minor interface
change if the download is judged unsafe or suspicious, requires user to
confirm/deny download)}
