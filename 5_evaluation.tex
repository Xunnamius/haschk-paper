\section{Evaluation} \label{sec:evaluation}

The primary goal of any \SYSTEM{} implementation is to alert end-users when the
resource they've downloaded is something other than what they were expecting. We
tested the effectiveness of our approach using the \SYSTEM{} extension for
Google Chrome, a real-world deployment of HotCRP, and a random sampling of
papers published in previous \CONFERENCE{} proceedings.

\subsection{Threat Model and Assumptions}

Include threat model for Chrome extension and FTP patch.

\subsection{Real-World Resource Corruption Detection with Google Chrome and HotCRP}

It also seems from ACME that HTTP challenges are good enough of a proof to issue
TLS certificates, so why not good enough for checksums? Threat model of ACME
thoroughly goes through this \cite{draft-ACME}.

\subsection{Deployment and Scalability}

Discuss envisioned deployment strategies for resource providers.

Can this be scaled? Yes it can. What are the practical limits? EDNS0 means it
ain't DNS size, though packet fragmentation is still a concern. How about max
record length? Maximum number of records? A service could have thousands or
millions of files it serves! Can DNS handle that? DHT failover is still a
solution anyway.

\subsection{Performance Overhead}

Additional Download Latency, Additional Network Load, Runtime overhead, etc. All
nixed.
