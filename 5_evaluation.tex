\section{Evaluation} \label{sec:evaluation}

The primary goal of any DNSCHK implementation is to alert end-users when the
resource they've downloaded is something other than what they were expecting. We
tested the effectiveness of our approach using the DNSCHK extension for Google
Chrome, a real-world deployment of HotCRP, and a random sampling of papers
published in previous \CONFERENCE{} proceedings.

\subsubsection{Threat Model and Assumptions}

tools.ietf.org/html/rfc3552 section-5

Include threat model for Chrome extension and FTP patch.

\subsection{Real-World Resource Corruption Detection with Google Chrome and HotCRP}

It also seems from ACME that HTTP challenges are good enough of a proof to issue
TLS certificates, so why not good enough for checksums? Threat model of ACME
thoroughly goes through this, cite bit.ly/2PYS30a

\subsection{Overhead}

Additional Download Latency, Additional Network Load, Runtime overhead, etc. All
nixed.
