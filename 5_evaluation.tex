\section{Evaluation} \label{sec:evaluation}

The primary goal of any \SYSTEM{} implementation is to alert end-users when the
resource they have downloaded is something other than what they were expecting.
We tested the effectiveness of our approach using the \SYSTEM{} extension for
Google Chrome, a real-world deployment of HotCRP, and a random sampling of
papers published in previous \CONFERENCE{} proceedings.

\subsection{Threat Model and Security Considerations}

\subsubsection{Compromised Resource}

We consider the case where an adversary can influence or even completely control
the victim's resource distribution mechanism (web page, file server, CDN, etc)
in any way. In this context, the adversary can trick the user into downloading a
compromised resource of the adversary's choice. This can be accomplished by
compromising the resource on the victim's system or tricking the user into
downloading a compromised resource on the adversary's remote system.

In this case, the adversary does not have control over any DNS zone(s) relevant
to the function of \SYSTEM{}.

If the adversary does not alter the Resource Identifier, the compromised
resource will fail integrity validation during the NAC Validation step.

If the adversary does alter the Resource Identifier, there are two
possibilities: a) the new Resource Identifier \textit{does not} exist in the DNS
zone, in which case \SYSTEM{} will fail to resolve the NAC, hence the NAC
Validation step will fail silently; b) the new Resource Identifier \textit{does}
exist in the DNS zone, therefore the ``new'' RI must be pointing to a different
resource's checksum. Unless the adversary's goal is to swap one or more resources
protected by \SYSTEM{} and a particular DNS zone with another resource also
protected by \SYSTEM{} and in that same zone, the NAC Validation step will fail.
For the aforementioned ``swap'' to work, the adversary would be required to both
change the RI and also offer to the victim the \SYSTEM{} protected resource the
``new'' RI corresponds to, which shrinks the attack surface significantly.

We note that, in our proof-of-concept implementation, the ``silent failure''
above allows \SYSTEM{} to be immediately deployed without risk of annoying the
user with false negatives. See \secref{discussion} for further discussion.

\subsubsection{Compromised Authoritative Checksum}

We consider the case where an adversary can completely control the victim DNS
zone(s) that allow \SYSTEM{} to function. Therefore, the adversary can return an
authoritative response of their choice to any DNS query.

In this case, the adversary does not have control over the victim's resource
distribution mechanism (web page, file server, CDN, etc).

DNSSEC ensures the validity and authenticity of DNS responses. In order for the
adversary to control any relevant DNS zones, they must have access to the
authoritative DNS server and/or the appropriate DNSSEC keys.

Even if the adversary achieved this level of compromise, they do not have the
ability to deliver a malicious payload in this case. However, the adversary
could use control over the relevant DNS zones to cause denial-of-service style
attacks against those attempting to download the resource by causing all NAC
Validation checks to fail. This is mitigated by \SYSTEM{} allowing the user to
``override'' its error states; \ie, \SYSTEM{} does not (and cannot, thanks to
the Chrome/WebExtensions API) mutate a downloaded resource. See
\secref{discussion} for further discussion on limitations due to the
Chrome/WebExtensions API.

\subsubsection{Compromised Resource and Authoritative Checksum}

We consider the case where an adversary can influence or even completely control
the victim's resource distribution mechanism (web page, file server, CDN, etc)
in any way. Additionally, the adversary can completely control the victim DNS
zone(s) that allow \SYSTEM{} to function. Therefore, the adversary can make the
user download a compromised resource and also return a (compromised) AC that
legally corresponds to said compromised resource.

\subsubsection{Determining the Origin Domain}

If an adversary manages to compromise a web page/server, they have two options.
They can mutate the resource directly, which would be observable via \SYSTEM{}.
They could also mutate the web/download page itself, replacing the anchor with a
malicious one that points to a compromised resource on the adversary's remote
system. \SYSTEM{} will still catch this due to the Chrome API's
\texttt{DownloadItem::referrer} property (distinct from the concept of HTTP
referrer).

However, a clever attacker can trick the Chrome API into populating
\texttt{DownloadItem::referrer} by redirecting the user to a valid and innocuous
page that very quickly redirects the user again to a compromised resource with
the goal of tricking the Chrome API into supplying \SYSTEM{} with a chosen
\texttt{DownloadItem::referrer}.

In order to prevent such implementation-specific attacks, we make a distinction
between the domain that the hyperlink containing the desired resource references
and the Origin Domain (OD)---or the domain of the document within which said
hyperlink exists. The extension is implemented such that the OD is resolved as
early as possible in the page loading process. The scope of the OD is at the tab
level, meaning there is one OD determined for each open browser tab. Once
determined for a tab, the OD should not be recalculated for some period of time.
If the tab is navigated and a download is started within a chosen time window,
the user will be asked to verify that the OD is what they expect it to be
(should be a familiar URL).

We catch potential redirection attacks by assuming any page that begins a
download in under three (3) seconds is suspicious and requires affirmation by
the user. We expect this extreme edge case to occurrence very rarely, if at all.

\subsection{Real-World Resource Corruption Detection with Google Chrome and HotCRP}

\TODO{It also seems from ACME that HTTP challenges are good enough of a proof to
issue TLS certificates, so why not good enough for checksums? Threat model of
ACME thoroughly goes through this~\cite{draft-ACME}.}

\TODO{We carefully evaluate the security, performance,
and practicality of \SYSTEM{} through implementing extensions in both a web
browser (Chrome) and FTP client (FileZilla); implementations are tested versus
common resource integrity violations. We find that \SYSTEM{} is more effective
than existing solutions, detects a wide variety of real-world integrity errors
across a diverse set of platforms, and is a scalable and immediately deployable
solution.}

\subsection{Deployment, Scalability, and Robustness}

\TODO{Discuss envisioned deployment strategies for resource providers.}

\TODO{Can this be scaled? Yes it can. What are the practical limits? EDNS0 means
it is not DNS size, though packet fragmentation is still a concern. How about
max record length? Maximum number of records? A service could have thousands or
millions of resources it serves! Can DNS handle that? DHT failover is still a
solution anyway.}

\subsection{Performance Overhead}

\TODO{Additional Download Latency, Additional Network Load, Runtime overhead,
etc. All nixed.}

While evaluating \SYSTEM{}, we observed no discernible additional network load
or CPU usage with the extension loaded into Chrome. As \SYSTEM{} does not
interrupt or manipulate resources as they are being downloaded, there is no
additional download latency introduced by \SYSTEM{}. 

Not computationally intensive. Requests are small, low network load. No
discernible performance overhead.
