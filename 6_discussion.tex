\section{Discussion} \label{sec:discussion}

In this section, we examine current and previous DNS-based and other
cryptographic schemes, most of which are based on public key cryptography.
Further, we note PGP's limiting human factors, how those factors also apply to
the checksum solution, and how the \SYSTEM{} solution avoids them. Thereafter,
we discuss some limitations of the \SYSTEM{} methodology, implementation, and
DNS itself.

\subsection{Additional Related Work}

\noindent\textbf{Cryptographic Data in DNS Resource Records.}    Storing
cryptographic data in the DNS network is not a new idea. The DNS-Based
Authentication of Named Entities (DANE) specification~\cite{DANE1, DANE2, DANE3}
defines the ``TLSA'' and ``OPENPGPKEY'' DNS resource records to store
cryptographic data. These resource record types, along with
``CERT''~\cite{CERT}, ``IPSECKEY''~\cite{IPSECKEY}, those defined by DNS
Security Extensions (DNSSEC)~\cite{DNSSEC}, and others demonstrate that storing
useful cryptographic data retrievable through the DNS network is feasible at
scale~\cite{DANE3, DANE1}. With \SYSTEM{}, however, we use ``TXT'' records to
map resource identifiers to authoritative hashes. In accordance with RFC
5507~\cite{RFC5507}, an actual \SYSTEM{} implementation would necessitate the
creation of a new DNS resource record type. \\

\noindent\textbf{PGP/OpenPGP.}    Though PGP addresses a fundamentally different
threat model than \SYSTEM{}, it is useful to note: many of the same human and UX
factors that make the cryptographically solid OpenPGP standard and its various
implementations so unpleasant for end users also exist in the context of
download integrity verification and checksums. End users of all backgrounds
cannot and \textit{will not} be burdened with manually verifying a checksum; as
was the case with PGP 5.0~\cite{PGPBad}, some users are likely confused by the
very notion of a checksum, if they are aware of checksums at all. As with all
mechanisms that further complicate a task, even users capable of manually
verifying the checksum of a file they've downloaded . To assume otherwise (see:
\secref{background}).

\noindent\textbf{Link Fingerprints and Subresource Integrity (SRI).}    (about)~\cite{LF, SRI}

\noindent\textbf{Content-MD5 HTTP Header.}    (about)~\cite{MD5Header}

\subsection{Limitations}

\subsubsection{DNSSEC Adoption is Slow}

DNSSEC is hard to configure correctly~\cite{DNSSEC-is-hard-1, DNSSEC-is-hard-2,
DNSSEC-is-hard-3, DNSSEC-is-hard-4}. However, it does not make the DNS network,
i.e. properly configured DNS servers protected with DNSSEC, any more vulnerable
to amplification or other types of reflection attacks~\cite{Ariya} than it
already is as a UDP-based content delivery service~\cite{USCERT, Vixie}. DNSSEC
does arguably make services that rely on it significantly more fragile because
DNSSEC is hard to configure correctly.

Further, DNSSEC adoption across is small and slow. Worldwide, less
than 14 percent of DNS requests have DNSSEC validated by the
resolver~\cite{APNIC} but thanks to community initiatives is on the
rise~\cite{Cloudflare} (use graph as figure from bit.ly/2zSR7A6).

\TODO{From SO: DNSSEC does have risks! It's hard to use, and harder to use
correctly. Often it requires a new work flow for zone data changes, registrar
management, installation of new server instances. All of that has to be tested
and documented, and whenever something breaks that's related to DNS, the DNSSEC
technology must be investigated as a possible cause. And the end result if you
do everything right will be that, as a zone signer, your own online content and
systems will be more fragile to your customers. As a far-end server operator,
the result will be, that everyone else's content and systems will be more
fragile to you. These risks are often seen to outweigh the benefits, since the
only benefit is to protect DNS data from in-flight modification or substitution.
That attack is so rare as to not be worth all this effort. We all hope DNSSEC
becomes ubiquitous some day, because of the new applications it will enable. But
the truth is that today, DNSSEC is all cost, no benefit, and with high risks.
The overwhelming majority of domain name zone administrators appear to be just
not aware of DNSSEC, or, even if they want to sign their zone, they cannot
publish a signed zone because of limitations in the service provided by the
registrar, or if they are aware and could sign their zone, then they don’t
appear to judge that the perceived benefit of DNSSEC-signing their zone
adequately offsets the cost of maintaining the signed zone.}

\subsubsection{DNS-Specific Protocol Limitations}

DNS~\cite{DNS1} was not originally designed to transport or store relatively
large amounts of data, though this has been addressed with EDNS0~\cite{EDNS}.
The checksums stored in DNS shouldn't be much longer than 128 bytes or the
output of the SHA512 function. Regardless, DNS resource record extensions exist
that store much more than 128 bytes of data~\cite{CERT, IPSECKEY, DANE3, DANE1}.

Several working groups are considering DNS as a storage medium for
checksums/hash output as well, such as securitytxt~\cite{draft-sectxt}. A widely
deployed example of DNS ``TXT'' resource records being used this way is
DKIM~\cite{DKIM}. \SYSTEM{} does not add to the danger of amplification and
other reflection attacks on DNS; these are generic DNS issues addressable at
other layers of the protocol.

\subsubsection{Chrome Implementation}

Our current JavaScript proof-of-concept implementation, as a Chrome extension,
isn't allowed to touch the resource file downloaded by Chrome and so can't
prevent the potentially-malicious resource file from being executed by the end
user—a feature Chrome/Chromium reserves for its own internal use. The Chrome
\textit{app} API~\cite{AppAPI} might have been of assistance as it allowed for
some limited filesystem traversal via a now deprecated native app API; there is
also a non-standard HTML5/WebExtensions FileSystem API that would provide
similar functionality were it to be widely considered~\cite{deadSpec}.

\SYSTEM{} would be even more effective as a browser extension if Chrome/Chromium
or the WebExtensions API allowed for an explicit \texttt{onComplete} event hook
in the downloads API. This hook would fire immediately before a file download
completed and the file became executable, \ie had its \texttt{.crdownload} or
\texttt{.download} extension removed. The hook would consume a
\texttt{Promise}/\texttt{AsyncFunction} that kept the download in its
non-complete state until said \texttt{Promise} completed. This would allow
\SYSTEM{}'s background page to do something like alter the download's
\texttt{DangerType} property and alert the end user to the dangerous download
naturally. This would have the advantage of communicating intent through the
browser's familiar UI and preventing the potentially-malicious download from
becoming immediately executable. Unfortunately, the closest the
Chrome/WebExtensions API comes to allowing \texttt{DangerType} mutations is the
\texttt{acceptDanger} method on the downloads API, but it is not suitable for
use with \SYSTEM{} as it is a background page based extension.

\subsection{Future Work}

(adapt wiki entry)
