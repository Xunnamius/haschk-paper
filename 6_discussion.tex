\section{Discussion} \label{sec:discussion}

\subsection{Related Work}

\textbf{DNS-Based Authentication of Named Entities (DANE)}

The DNS-Based Authentication of Named Entities (DANE) specification [RFC6698]
introduces the DNS "TLSA" resource record (RR) type ("TLSA" is not an acronym).
TLSA records associate a certificate or a public key of an end-entity or a
trusted issuing authority with the corresponding Transport Layer Security (TLS)
[RFC5246] or Datagram Transport Layer Security (DTLS) [RFC6347] transport
endpoint. DANE relies on the DNS Security Extensions (DNSSEC) [RFC4033]. DANE
TLSA records validated by DNSSEC can be used to augment or replace the use of
trusted public Certification Authorities (CAs).

DANE + OpenPGP: tools.ietf.org/html/rfc7929

Different concern. tools.ietf.org/html/rfc7671

\textbf{PGP}

Notoriously hard to use. Similar problems as checksums and DNSSEC in that its
hard to use correctly. Solves a fundamentally different problem in a different
domain (email).

\subsection{Deployment and Scalability}

Discuss envisioned deployment strategies for resource providers.

Can this be scaled? Yes it can. What are the practical limits? EDNS0 means it
ain't DNS size, though packet fragmentation is still a concern. How about max
record length? Maximum number of records? A service could have thousands or
millions of files it serves! Can DNS handle that? DHT failover is still a
solution anyway.

\subsection{Limitations}

\textbf{DNSSEC Adoption is Slow}

DNSSEC is hard to use and harder to use correctly. It does not make the DNS
network, i.e. properly configured DNS servers protected with DNSSEC, any more
vulnerable to reflection (cite Neustar) and amplification (cite
ieeexplore.ieee.org/document/4159821) attacks than it already is as a UDP-based
content service (cite ss.vix.su/~vixie/isc-tn-2012-1.txt,
us-cert.gov/ncas/alerts/TA14-017A) but does arguably make services significantly
more fragile because DNSSEC is hard to get right.

The metrics of signed DNSSEC zones are not easy to come by for the entire
Internet (apnic). DNSSEC adoption across is small and slow. Worldwide, less than
14 percent of DNS requests have DNSSEC validated by the resolver (cite apnic)
but thanks to community initiatives is on the rise (cite
blog.cloudflare.com/automatically-provision-and-maintain-dnssec) (use graph as
figure bit.ly/2zSR7A6).

From SO: DNSSEC does have risks! It's hard to use, and harder to use correctly.
Often it requires a new work flow for zone data changes, registrar management,
installation of new server instances. All of that has to be tested and
documented, and whenever something breaks that's related to DNS, the DNSSEC
technology must be investigated as a possible cause. And the end result if you
do everything right will be that, as a zone signer, your own online content and
systems will be more fragile to your customers. As a far-end server operator,
the result will be, that everyone else's content and systems will be more
fragile to you. These risks are often seen to outweigh the benefits, since the
only benefit is to protect DNS data from in-flight modification or substitution.
That attack is so rare as to not be worth all this effort. We all hope DNSSEC
becomes ubiquitous some day, because of the new applications it will enable. But
the truth is that today, DNSSEC is all cost, no benefit, and with high risks.

The overwhelming majority of domain name zone administrators appear to be just
not aware of DNSSEC, or, even if they want to sign their zone, they cannot
publish a signed zone because of limitations in the service provided by the
registrar, or if they are aware and could sign their zone, then they don’t
appear to judge that the perceived benefit of DNSSEC-signing their zone
adequately offsets the cost of maintaining the signed zone.

\textbf{DNS-Specific Protocol Limitations}

Whether we created our own DNS resource record type or settled for a well-known
TXT record, there are other DNS records that could be far larger. Resource
records like CERT, IPSECKEY, OPENPGPKEY, and TLSA (with a full certificate) may
hold several kilobytes of data, far more than any compliant security.txt file
ever would.

I do hope that all the major DNS server operators have implemented DNS-over-TCP
or some sort of rate limiting by now, but I feel that's out of the scope of this
project. All we can do is urge any company running a vulnerable DNS server to
quickly create a security.txt file.

DNS data can handle a lot more then the amount of data needed here. And
amplification protection can be done simply by using TCP or insisting on
DNS-COOKIES for these answers. This is a generic DNS problem is solved by DNS
people and there is no need for additional requirements here.

\subsection{Future Work}

\TODO{(adapt wiki entry)}
