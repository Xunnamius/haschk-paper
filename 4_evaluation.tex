\section{Evaluation} \label{sec:evaluation}

The primary goal of any \SYSTEM{} implementation is to alert end-users when the
resource they have downloaded is something other than what they were expecting.
In this section, we evaluate our approach by first assessing the threat model
\SYSTEM{} addresses, followed by an examination of our proof-of-concept Google
Chrome extensions, \DNSSYS{} and \DHTSYS{}. Specifically, we evaluate the
obstacles to scalability and potential performance overhead of our extensions.
We then test our implementations versus a real-world deployment of HotCRP and/or
a random sampling of papers published in previous \CONFERENCE{} proceedings.

\subsection{The Threat Model}

\subsubsection{Compromised Resource}

We consider the case where an attacker can influence or even completely control
the victim's resource distribution mechanism (web page, file server, CDN, etc)
in any way. In this context, the attacker can trick the user into downloading a
compromised resource of the attacker's choice. This can be accomplished by
compromising the resource on a victim system or tricking the user into
downloading a compromised resource on the attacker's remote system.

In either case, the attacker does not have control over the backend system
responsible for mapping RIs to ACs relevant to the function of \SYSTEM{}.

If the attacker does not alter the RI, the compromised resource will fail
integrity validation during the NAC Validation step.

If the attacker does alter the RI, there are two possibilities: a) the new RI
\textit{does not} exist in the backend, in which case \SYSTEM{} will fail to
resolve the NAC, hence the NAC Validation step will fail silently; b) the
``compromised'' RI \textit{does} exist in the backend, therefore the RI must be
pointing to a different resource's checksum. Unless the attacker's goal is to
swap one or more resources protected by \SYSTEM{} and a particular backend with
another resource also protected by \SYSTEM{} and that backend, the NAC
Validation step will fail. For such a ``swap'' to work, the attacker would be
required to both change the RI and also offer to the victim the
\SYSTEM{}-protected resource the ``compromised'' RI corresponds to, which
shrinks the attack surface here significantly.

\subsubsection{Compromised Authoritative Checksum}

We consider the case where an attacker can completely control the victim highly
available backend that allows \SYSTEM{} to function. Therefore, the attacker can
return an authoritative response of their choice to any query.

In this case, the attacker does not have control over the victim's resource
distribution mechanism (web page, file server, CDN, etc).

Even if the attacker achieved this egregious level of compromise, they do not
have the ability to deliver a malicious payload in this case. However, the
attacker could use control over the relevant backends to cause denial-of-service
style attacks against those attempting to download the resource by causing all
NAC Validation checks to fail. This is mitigated by \SYSTEM{} allowing the user
to ``override'' its error states; \ie \SYSTEM{} does not mutate or quarantine
downloaded resources. See \secref{discussion} for further discussion on
limitations due to the Chrome/WebExtensions API.

\subsubsection{Compromised Resource and Authoritative Checksum}

We consider the case where an attacker can influence or even completely control
the victim's resource distribution mechanism (web page, file server, CDN, etc)
in any way. Additionally, the attacker can completely control the victim highly
available backend that allows \SYSTEM{} to function. Therefore, the attacker can
make the user download a compromised resource and also return a (compromised) AC
that legally corresponds to said compromised resource.

\subsection{Real-World Resource Corruption Detection}

\PUNT{As described in the IETF ACME draft specification, HTTP and DNS based
authentication challenges are a good enough basis to issue TLS certificates for
services such as the popular TLS Certificate Authority \emph{Let's
Encrypt}~\cite{draft-ACME}.}

To further evaluate the effectiveness of our mitigation, we tested our
proof-of-concept \SYSTEM{} Chrome extensions implementation against a series of
common real-world resource integrity violations. The impetus behind any such
resource integrity SCA is to have the resource pass through undetected by
abusing the trust between client and provider with the hope that an unsuspecting
user will interact with it.

\subsubsection{\DNSSYS{}}

As a test bed, we launched a heavily modified version of the popular open source
research submission and peer review software, \emph{HotCRP}. Our modifications
allowed us to interactively corrupt submissions and manipulate relevant DNS
entries. We uploaded a variety of different resources, including legitimate
PDFs, to our HotCRP instance.

We then selectively corrupted half of these resources with random bit flips.
Interesting to note: the HotCRP instance did not update the checksums of the
resources we corrupted. The only way to notice the attack would be to manually
validate PDFs downloaded from HotCRP after the fact, something most users are
not willing to do. Further, leveraging our administrator access to the instance,
we manually uploaded files known to be malicious, replacing previously submitted
innocuous PDFs in the HotCRP backend. We also uploaded resources without
including them in our backend file.

When we attempted to download the corrupted resources, including those uploaded
manually via side channel, \SYSTEM{} correctly flagged each corrupted resource.
Unmodified resources were not flagged and so were not affected; this includes
the resources whose ACs we purposely excluded from our backend file.

\subsubsection{\DHTSYS{}}

Further, we implemented a ``redirection'' attack wherein a ``compromised'' PHP
script added to the HotCRP instance redirected users several times before
quickly triggering the download of a corrupted resource from a disparate domain.
\SYSTEM{} correctly flagged this download as suspicious once the download began,
successfully warning the user.

We conclude that our proof-of-concept \SYSTEM{} implementation and approach are
more effective than existing approaches at detecting integrity violations in
arbitrary resources on the internet at scale; this is evident especially when
\SYSTEM{} is compared to the de facto standard, checksums.

\subsection{Obstacles to Scalability, Deployment}

As \SYSTEM{} is predicated on a distributed authenticated highly-available
backend, we conclude that the scalability of \SYSTEM{} can be reduced to the
scalability of its backend. We are aware of no other obstacles to scalability
beyond those inherited from the underlying backend system.

In respect to DNS specifically, packet fragmentation can be a concern for high
performance networks~\cite{EDNS}, but this is an artifact of DNSSEC and related
protocols rather than \SYSTEM{}~\cite{DNSSEC}. Further, we are aware of no
practical limits or protocol-based restrictions on the scalability of a backend
file itself or its sub-zones. A service can host tens of thousands of resource
records in their backend file~\cite{DNS1, DNS2}.

With the HotCRP demo, the totality of our resource deployment scheme consisted
of the addition of a new TXT entry to our backend file---accomplished via API
call---during HotCRP's paper submission process. This new TXT entry consisted of
a mapping between a RI and its corresponding AC.

We find a DNS record addition or update during the resource deployment process
to be simple enough for service administrators to implement and presents no
significant burden to deployment outside of DNS API integration into a
development team or other entity's software deployment toolchain. For reference,
we implemented the functionality that automatically adds (and updates) the DNS
records mapping the ACs and RIs of papers uploaded to our HotCRP instance in
under 10 lines of JavaScript.

We note that, in the case where an entity's content distribution mechanism
relies on, for instance, a mirroring service, third-party CDN, et cetera
\emph{that randomly or disparately mangles resource URL paths}, our
proof-of-concept implementations currently require each "mangled" RI permutation
representing a resource to be added to the backend, even if they all represent
the same resource by a different name/path. This is discussed further in the
context of URNs in \secref{discussion}.

\subsection{Performance Overhead}

While evaluating our \SYSTEM{} implementations, we observed no additional
non-trivial network load or CPU usage with the extension loaded into Chrome.
Measurements were taken using the Chrome developer tools. Intuitively, this
makes sense since our \SYSTEM{} implementations make at most two queries to the
backend before rendering a judgement. Hence, we find that \SYSTEM{} introduces
no additional performance overhead to function. Further, as our implementations
of \SYSTEM{} do not interrupt or manipulate resources as they are being
downloaded, there is no additional download latency introduced by \SYSTEM{}.

\subsection{Attacking OD Resolution in the Browser}

While determining ODs, a clever attacker might attempt to fool this process by
redirecting users one or more times before triggering a direct download of a
compromised resource. The redirects would allow an attacker to completely
manipulate the OD, with the ability to trick an unsuspecting user into
downloading a compromised resource with valid entries in the backend system.

We mitigate this threat by leveraging JavaScript's document-wide \emph{trusted
event}~\cite{TrustedEvents} delegation capability. Specifically, when a tab
navigation event is observed, the tab is flagged \texttt{suspicious} by default
and the determined OD is not updated (\ie it remains at its previous value). If
the user interacts with the tab (\ie a trusted click or key press event is
observed after navigation completes), the \texttt{suspicious} flag is cleared
and the OD is updated. If another tab navigation event occurs without user
interaction first (\eg a quick redirect), this process repeats recursively. If a
download is observed coming from a tab flagged \texttt{suspicious}, the user is
warned about the suspicious circumstances similarly to receiving an ``unsafe''
judgement.

While this mitigates the attack as described, it has the side effect of
potentially generating false positive warnings when 1) the user is redirected to
a legitimate website---such as a download mirroring service---that automatically
triggers a download after some amount of time when also 2) the user does not
interact with the page at all before the download begins. We argue such cases
are non-average and the tradeoff here is worthy.
