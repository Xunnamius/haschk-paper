\section{Security Goals and Threat Model} \label{sec:model}

The goal of \SYSTEM{} is to ensure the integrity of arbitrary resources
downloaded over the internet, even in the case where the system hosting those
resources is completely compromised. Given a \SYSTEM{}-aware client and a
resource provider with a conforming backend, the user must be automatically
warned whenever what they are downloading does not match a corresponding
authoritative checksum. Given \SYSTEM{} is properly deployed, this must function
without false negatives, \ie{compromised resources that do not trigger a
warning}, or false positives, \ie{benign resources that do trigger a warning}.

Our model considers (1) a \SYSTEM{}-aware client as the \emph{frontend}, (2) a
\emph{server} or servers as the distribution system co-hosting a specific
\emph{resource} and an HTML file (\emph{web page}) with a hyperlink pointing to
that resource, and (3) a separate high availability system implementing
\SYSTEM{} as the \emph{backend}. Specifically, we consider a \emph{generic
frontend} that might be any web-facing software, such as FileZilla, and a
\emph{browser frontend} such as Google Chrome or Mozilla Firefox.

When considering a generic frontend, we assume the adversary can make the server
respond to a resource request from the frontend with any resource, including a
compromised version of a resource. We also assume the adversary can tamper with
any other server response, including adding, removing, or manipulating one or
more checksums.

When considering a browser frontend and web server, we assume the user first
navigates to a web page on the server and then clicks a hyperlink to download a
resource hosted either on the same server or externally on a third party server
such as a CDN. We also assume the adversary can manipulate the resource wherever
it is stored, and can further manipulate the web page co-hosting the resource's
hyperlink alongside the resource's authoritative checksum. The adversary might
compromise the resource; modify the hyperlink to point to a compromised resource
anywhere on the internet; and/or add, remove, or manipulate any checksums on the
web page corresponding to the resource.

Hence, we do not trust the integrity of the server's responses in any scenario.
We do, however, trust the integrity of the backend's responses. Further, we
expect the backend to be highly available. In \secref{evaluation}, we go over
the implications of a compromised backend.
