\section{Threat Model and Security Goals} \label{sec:model}

\subsection{Compromised Resource}

We consider the case where an attacker can influence or even completely control
the victim's resource distribution mechanism (web page, file server, CDN, etc)
in any way. In this context, the attacker can trick the user into downloading a
compromised resource of the attacker's choice. This attack can be accomplished
by compromising the resource on a victim system or tricking the user into
downloading a compromised resource on the attacker's remote system.

In either case, the attacker does not have control over the backend system
responsible for mapping RIs to ACs relevant to the function of \SYSTEM{}.

If the attacker does not alter the RI, the compromised resource will fail
integrity validation during the NAC Validation step.

If the attacker does alter the RI, there are two possibilities: (1) the new RI
\textit{does not} exist in the backend, in which case \SYSTEM{} will fail to
resolve the NAC, hence the NAC Validation step will fail or (2) the
``compromised'' RI \textit{does} exist in the backend, therefore the RI must be
pointing to a different resource's checksum.

In the first case, there are two further possibilities: a) NAC validation fails
and \SYSTEM{} \emph{is not} in strict mode, so a ``neutral'' judgment is
rendered; b) NAC validation fails and \SYSTEM{} \emph{is} in strict mode, so an
``unsafe'' judgement is rendered, warning users that the resource is likely
compromised. \TODO{This seems to contradict the earlier claims of compromising
convenience and favoring security. Shouldn't it fail if there is any
possibility of compromise? Also, even if you don't want to do that, I would
suggest that you rename strict to default and not strict to something else, like
unsafe mode.}

In the second case, unless the attacker's goal is to swap one or more resources
protected by \SYSTEM{} and a particular backend with another resource also
protected by \SYSTEM{} and the same backend, the NAC Validation step will fail.
For such a ``swap'' to work, the attacker would be required to both change the
RI and also offer to the victim the \SYSTEM{}-protected resource the
``compromised'' RI corresponds to, which shrinks the attack surface here
significantly. \TODO{This deserves one or two more sentences of explanation:
what exactly does the attacker have to do to succeed in this scenario?}

\subsection{Compromised Authoritative Checksum}

We consider the case where an attacker can completely control the high
availability backend that allows \SYSTEM{} to function. In this context, the
attacker can return an authoritative response of their choice to any query.

In this case, the attacker does not have control over the victim's resource
distribution mechanism (web page, file server, CDN, etc).

Even if the attacker achieved this egregious level of compromise, they do not
have the ability to deliver a malicious payload in this case. However, the
attacker could use control over the relevant backends to cause denial-of-service
style attacks against those attempting to download the resource by causing all
NAC Validation checks to fail. This is mitigated by \SYSTEM{} allowing the user
to ``override'' its error states; \ie \SYSTEM{} does not mutate or quarantine
downloaded resources. \TODO{Get rid of this?}.

\subsection{Compromised Resource and Authoritative Checksum}

We consider the case where an attacker can influence or even completely control
the victim's resource distribution mechanism (web page, file server, CDN, etc)
in any way. Additionally, the attacker can completely control the victim high
availability backend that allows \SYSTEM{} to function. Therefore, the attacker
can make the user download a compromised resource and also return a
(compromised) AC that legally corresponds to said compromised resource.
\TODO{Feels incomplete.}
