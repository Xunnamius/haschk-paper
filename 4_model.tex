\section{Security Goals and Threat Model} \label{sec:model}

The goal of \SYSTEM{} is to ensure the integrity of arbitrary resources
downloaded over the internet, even in the case where the system hosting those
resources is completely compromised. Given a \SYSTEM{}-aware client and a
resource provider with a conforming backend, the user must be automatically
warned whenever what they are downloading does not match a corresponding
authoritative checksum. Given proper deployment, \SYSTEM{} should, to whatever
extent possible, operate without issuing false negatives, \ie{ compromised
resources that do not trigger a warning}, or false positives, \ie{ benign
resources that do trigger a warning}.

We consider the following entities: (1) a \SYSTEM{}-aware client as the
\emph{frontend}, (2) a \emph{server} (or servers) as the distribution system
hosting the \emph{resource}, and (3) a separate high availability system
implementing \SYSTEM{} as the \emph{backend}. Specifically, we consider a
\emph{generic frontend} that might be any web-facing software, such as
FileZilla, and a \emph{browser frontend} such as Google Chrome or Mozilla
Firefox.

For a generic frontend, we assume the adversary can make the server respond to a
resource request with any resource, including a compromised version of a
resource. We also assume the adversary can tamper with any other server
response, including adding, removing, or manipulating one or more checksums if
they appear.

For a browser frontend and web server, we contemplate two scenarios. From either
an existing tab/window or a new one, the user (1) directly enters the resource
URI into the browser, initiating a direct download or (2) navigates to an HTML
file (\emph{web page}) on the server containing a hyperlink pointing to the
resource. The user clicks this hyperlink to download the resource. This resource
is hosted on the same server as the web page or externally on a third party
server such as a CDN. Both are valid deployments in the context of \SYSTEM{}. We
assume the adversary can manipulate the resource and web page wherever they are
stored. The adversary may: (1) compromise the resource, (2) modify the hyperlink
to point to a compromised resource anywhere on the internet, and/or (3) add,
remove, or manipulate any checksums that appear on the web page or elsewhere.

Hence, we do not trust the integrity of the server's responses in any scenario.
We do, however, trust the integrity and authenticity of the backend's responses.
Further, we expect the backend to be highly available. In \secref{evaluation},
we go over the implications of a compromised backend.

As an aside, we consider here the standard non-\SYSTEM{} case where the resource
is co-hosted alongside a web page containing both a hyperlink pointing to the
resource \emph{and an authoritative checksum corresponding to the resource}. As
demonstrated throughout the paper, when a server co-hosting resources and
checksums is compromised, it is trivial to manipulate both checksum and
resource, rendering all forms of checksum verification irrelevant. Assuming
proper deployment, it is the goal of \SYSTEM{} to ensure this is never the case.
