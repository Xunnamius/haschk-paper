%\documentclass[10pt,preprint,nocopyrightspace,nonatbib]{sigplanconf}
%\documentclass[9pt,preprint]{sig-alternate-no-permission}
%\documentclass[9pt,preprint]{sig-alternate}
%\documentclass[letterpaper,twocolumn,10pt]{article}
%\usepackage{usenix}
%\documentclass[pageno]{jpaper}
%\documentclass[10pt,preprint]{sigplanconf}
\documentclass[sigplan,10pt,anonymous]{acmart}\settopmatter{printfolios=true}
\usepackage{mathptmx} % This is Times font

\setcopyright{none}             %% For review submission
%\conferenceinfo{SOSP'15}{October 4--7, 2015, Monterey, CA}
%\copyrightyear{2015}
% The following \documentclass options may be useful:

% preprint      Remove this option only once the paper is in final form.
% 10pt          To set in 10-point type instead of 9-point.
% 11pt          To set in 11-point type instead of 9-point.
% authoryear    To obtain author/year citation style instead of numeric.

\usepackage{minted}
\usepackage{epsfig}
%\usepackage[utf8x]{inputenc}
\usepackage{algorithm}
\usepackage{amsmath, amssymb}
\usepackage[noend]{algpseudocode}
\usepackage{enumitem}      % adjust spacing in enums
%\usepackage{subfig}
\usepackage{caption}
\usepackage{subcaption}
\usepackage{multirow}
\usepackage{rotating}
\usepackage{wrapfig}
\usepackage{tabu}

\let\bibhang\relax
\let\citename\relax
\let\bibfont\relax
\let\citeauthor\relax
\let\Citeauthor\relax
\let\citefullauthor\relax
\let\citetext\relax
\let\defcitealias\relax
\let\citet\relax
\let\citep\relax
\let\Citep\relax
\let\Citealt\relax
\let\citealt\relax
\let\citealp\relax
\let\Citealp\relax
\let\Citet\relax

\expandafter\let\csname ver@natbib.sty\endcsname\relax

\DeclareCaptionFormat{subfig}{\figurename~#1#2#3}
\DeclareCaptionSubType*{figure}
\captionsetup[subfigure]{format=subfig,labelsep=colon,labelformat=simple}

\usepackage[natbib=true,backend=bibtex,firstinits=true,style=numeric-comp,sorting=nyt,defernumbers,maxnames=99,maxcitenames=99]{biblatex}
\usepackage{balance}
\usepackage{adjustbox}

\usepackage{pgfplots}
% options for pgfplots
\pgfplotsset{compat=1.8,compat/show suggested version=false}
\usetikzlibrary{plotmarks}
\usetikzlibrary{calc}
%\pgfplotsset{compat=newest}
\pgfplotsset{
   /pgfplots/bar  cycle  list/.style={/pgfplots/cycle  list={%
        {black,fill=black!30!white,mark=none},%
        {black,fill=red!30!white,mark=none},%
        {black,fill=green!30!white,mark=none},%
        {black,fill=yellow!30!white,mark=none},%
        {black,fill=brown!30!white,mark=none},%
     }
   },
}
% begin of externalization
\usetikzlibrary{external}
\tikzexternalize[prefix=out/]
\tikzexternalize
% don't externalize todonotes
%\makeatletter
%\renewcommand{\todo}[2][]{\tikzexternaldisable\@todo[#1]{#2}\tikzexternalenable}
%\makeatother
% end of externalization
\usetikzlibrary{patterns}
\usepgfplotslibrary{groupplots}
\pgfplotsset{
every axis label/.append style={font=\small},
tick label style={font=\small},
}
% options for paragraphs and lists
\setlist{noitemsep,topsep=0pt}
% options for biblatex
\bibliography{refs}
%\renewcommand{\bibfont}{\footnotesize}


\usepackage{fancyhdr}

% Ensure letter paper
\pdfpagewidth=8.5in
\pdfpageheight=11in

%%%%%%%%%%%---SETME-----%%%%%%%%%%%%%
\newcommand{\microsubmissionnumber}{440}
\newcommand{\asplossubmissionnumber}{440}
%%%%%%%%%%%%%%%%%%%%%%%%%%%%%%%%%%%%

\fancypagestyle{firstpage}{
%   \fancyhf{}
% \setlength{\headheight}{50pt}
% \renewcommand{\headrulewidth}{0pt}
%   \fancyhead[C]{\normalsize{ASPLOS 2018 Submission
%       \textbf{\#\asplossubmissionnumber} -- Confidential Draft -- Do NOT Distribute!!}} 
%   \pagenumbering{arabic}
}

\pagenumbering{arabic}


\setlength{\itemsep}{0pt}
%\setlength{\topsep}{0pt}
%\setlength{\partopsep}{0pt}
%\setlength{\parsep}{1pt}
%\setlength{\parskip}{1pt}
\setlength{\abovecaptionskip}{1pt plus 2pt minus 2pt}
\setlength{\textfloatsep}{5pt}
%\setlength{\bibitemsep}{0pt}

\sloppy

\newif{\ifanonymous}
\anonymoustrue

%\newcommand{\comment}[[1]{\textcolor{red}{#1}}
\newcommand{\myworries}[1]{\textcolor{red}{#1}}
\newcommand{\cutout}[1]{}
\newcommand{\smallcaption}[1]{\caption[#1]{{\protect\small \protect\bf #1}}}
\newcommand{\dids}{{\sc dids}}

\graphicspath{{./figs/}}
\date{}

\algnewcommand{\LineComment}[1]{\(\triangleright\) #1}

% some useful shortcuts
\newcommand{\ie}{\textit{i.e., }}
\newcommand{\eg}{\textit{e.g., }}
\newcommand{\CC}{C\nolinebreak\hspace{-.05em}\raisebox{.5ex}{\tiny\bf +}\nolinebreak\hspace{-.10em}\raisebox{.5ex}{\tiny\bf +}}

% units for results
\newcommand{\us}{\,$\mu$s}
\newcommand{\ms}{\,ms}
\newcommand{\KB}{\,KB}
\newcommand{\MB}{\,MB}
\newcommand{\GB}{\,GB}
\newcommand{\MHz}{\,MHz}
\newcommand{\GHz}{\,GHz}

% new latex commands:
%   Remove long section
\newcommand{\PUNT}[1]{}
\newcommand{\TABLETWO}[1]{}
%   Label work to be done
\definecolor{gray}{gray}{0.75}
\newcommand{\TODO}[1]{\textcolor{gray}{\textbf{\ [TODO:\ #1]\ }}}
\newcommand{\TR}[1]{#1}
%\newcommand{\TR}[1]{}
%\newcommand{\TODO}[1]{}
\newcommand{\FIX}[1] {\textcolor{red}{\textbf{\ [FIX:\ #1]\ }}}
%   Referencing various pieces of the document:
\newcommand{\figref}[1]{Fig.~\ref{fig:#1}}
\newcommand{\figsref}[2]{Figures~\ref{fig:#1} and~\ref{fig:#2}}
\newcommand{\figrref}[2]{Figures~\ref{fig:#1}--\ref{fig:#2}}
\newcommand{\secref}[1]{Section~\ref{sec:#1}}
\newcommand{\secsref}[2]{Sections~\ref{sec:#1} and~\ref{sec:#2}}
\newcommand{\eqnref}[1]{Eqn.~\ref{eqn:#1}}
\newcommand{\eqnsref}[2]{Equations~\ref{eqn:#1} and~\ref{eqn:#2}}
\newcommand{\eqnrref}[2]{Equations~\ref{eqn:#1}--\ref{eqn:#2}}
\newcommand{\insref}[1]{Instruction~\ref{ins:#1}}
\newcommand{\tblref}[1]{Table~\ref{tbl:#1}}
\newcommand{\appref}[1]{Appendix~\ref{app:#1}}

\newcommand{\algoref}[1]{Algorithm~\ref{algo:#1}}

% Custom hyphenation rules

%\DeclareMathOperator{\minimize}{minimize}
%\DeclareMathOperator{\st}{s.t.}
%\DeclareMathOperator*{\argmin}{arg\,min}
%\DeclareMathOperator*{\argmax}{arg\,max}
\newcommand{\argmin}{\arg\!\min}
\newcommand{\argmax}{\arg\!\max}
\newcommand{\minimize}{minimize}
\newcommand{\optimize}{optimize}
\newcommand{\ceil}[1]{\lceil #1 \rceil}
\newcommand{\floor}[1]{\lfloor #1 \rfloor}
\newcommand{\st}{s.t.}

\newcommand{\SYSTEM}{DNSCHK}

\pdfstringdefDisableCommands{
    \def\\{}
    \def\unskip{}
    \def\texttt#1{<#1>}
}

%-------------------------------------------------------------------------
\begin{document}

\date{}
\title[\SYSTEM{}]{\SYSTEM{}: \TODO{Needs After Colon Title}}

\author{
% Not used?
}

\begin{abstract}

  Downloading resources over the internet comes with many risks,
  including the chance that a malicious actor has replaced the
  resource you think you are accessing with a compromised version. The
  current standard for addressing this risk is the use of
  \emph{checksums} coupled with a secure transport layer: users
  download a resource and compare its checksum with a posted checksum
  from the developers to ensure a match. Among the many problems with
  the current use of checksums are (1) user apathy---for most users,
  hand-calculating the checksum and comparing to the published version
  are too tedious---and (2) co-hosting resources and checksums---a
  malicious actor who compromises a resource can trivially compromise
  a checksum a hosted using the same infrastructure. In this paper we
  propose \SYSTEM{}, a novel resource validation scheme meant as a
  complete replacement for current checksum based approaches.
  \SYSTEM{} automates the tedious parts of verification to eliminate
  user apathy while leveraging highly-available distributed systems to
  separate resources from digests, making these systems and their end
  users more resilient to resource integrity attacks.  We carefully
  evaluate the security, performance, and practicality of \SYSTEM{}
  through implementing extensions in both a web browser (Chrome) and
  FTP client (FileZilla); implementations are tested versus common
  resource integrity violations. We find that \SYSTEM{} is more
  effective than existing solutions, detects a wide variety of
  real-world integrity errors across a diverse set of platforms, and
  is a scalable and immediately deployable solution.

\end{abstract}


\maketitle
\thispagestyle{firstpage}
\pagestyle{plain}



%\vskip -100pt


%\category{C.1.3}{Other Architectural Styles}{Adaptable architectures}
%\category{I.2.8}{ Problem Solving, Control Methods, and Search}{Heuristic methods}
%\category{D.4.8}{Performance}{Measurements}
%\terms{Performance, Design, Experimentation}
%\keywords{Adaptive Systems, Power Management, Decision-tree}


\section{Introduction} \label{sec:introduction}

Using the internet to receive data is a remarkably simple and painless process
for application developers and end users alike. When using a browser such as
Google Chrome, this process can be summarized as: a user requests a server
resource at some URL, the server responds with the desired resource, and then
the browser completes downloading the resource.

Unfortunately, downloading content over the internet can be risky.

Supply Chain Attacks (SCA) are the compromise of software source code via cyber
attack, insider threat, or other attack on one or more phases of the development
and deployment life cycle. These attacks are made possible due to proximity and
have the goal of infecting and exploiting one or more victims---usually the
software company's customer base. \SYSTEM{} only protects against SCAs that
occur after the Authoritative Hash (AH) is calculated. AH calculation is
necessarily more likely to occur later in the software development life cycle or
very early in the deployment process. If an attacker is able to execute a
successful SCA before the AH is calculated, \SYSTEM{} would propagate the
compromised Authoritative Hash.

\TODO{(include table of supply chain stages that \SYSTEM{} is effective in)}

Although early Supply Chain Attacks are devastating, they are not the only
popular form of the attack. Many devastating supply chain attacks occur late in
the software development and deployment process, including \TODO{(choose three examples)}. Further, \TODO{(popular attacks at various levels, including CDNs)}.

Discuss ``free," \ie no interface changes, no addition to resource download
time, no additional burden on the end user (qualified statement).~\cite{DNSSEC}

In summary, our primary contributions are:

\begin{itemize}

  \item We propose a novel practical defense against receiving malicious,
  corrupted, or compromised resources over the internet. Contrasted with current
  solutions, our defense requires no source code or infrastructure changes at
  any level other than DNS, does not employ unreliable heuristics, does not
  interfere with other software or extensions that also handle resource
  downloads, and can be transparently deployed without adding to the
  \textit{fragility} of DNSSEC-enabled systems; it protects end users whose
  software implements \SYSTEM{} while remaining unnoticeable to users of whose
  software does not.

  \item We present our prototype \SYSTEM{} implementations for Google Chrome and
  FileZilla and demonstrate its effectiveness in automatically and transparently
  mitigating the accidental consumption of compromised resources from a
  compromised server hosting a compromised web portal. To the best of our
  knowledge, this is the \emph{first} system providing such capabilities with
  little implementation cost and at no cost to the end user. Further, we release
  the \SYSTEM{} solution to the community as open source software\footnote{The
  Chrome extension is available at https://tinyurl.com/dnschk-actual}.

  \item We carefully and extensively evaluate the security, scalability, and
  performance of our automated defense against resource corruption to
  demonstrate the effectiveness and high practicality of the \SYSTEM{} approach.
  Specifically, we find no obstacles to efficient scalability given choice of
  distributed system and no performance overhead compared to downloads without
  \SYSTEM{}. We further provide a publicly accessible demonstration of
  \SYSTEM{}'s utility via a patched HotCRP instance\footnote{The patched HotCRP
  instance is available at https://tinyurl.com/dnschk-hotcrp}.

\end{itemize}

\input{2_motivation}
\input{3_design}
\section{Implementation} \label{sec:implementation}

\subsection{DNSCHK: Google Chrome Extension}

\TODO{(describe implementation details; works with DNS or DHT and is published
to Chrome store; no interface changes!--i.e. downloads work exactly the same
with or without the extension; users still have to confirm/deny suspicious
judgements, but they're rare occurrences)}

\TODO{(should we provide a link to/description of the hotcrp
demo?)}

    \subsection{Expanded Threat Model}

    \TODO{(enumerate additional threats)}

\subsection{DNSCHK: FileZilla (FTP) Patch}

\TODO{(describe implementation details; minor interface change if the download
is judged unsafe or suspicious, requires user to confirm/deny download)}

    \subsection{Expanded Threat Model}

    \TODO{(enumerate additional threats)}

\section{Evaluation} \label{sec:evaluation}

The primary goal of any \SYSTEM{} implementation is to alert end-users when the
resource they have downloaded is something other than what they were expecting.
We tested the effectiveness of our approach using the \SYSTEM{} extension for
Google Chrome, a real-world deployment of HotCRP, and a random sampling of
papers published in previous \CONFERENCE{} proceedings.

\subsection{Threat Model and Security Considerations}

\subsubsection{Compromised Resource}

We consider the case where an adversary can influence or even completely control
the victim's resource distribution mechanism (web page, file server, CDN, etc)
in any way. In this context, the adversary can trick the user into downloading a
compromised resource of the adversary's choice. This can be accomplished by
compromising the resource on the victim's system or tricking the user into
downloading a compromised resource on the adversary's remote system.

In this case, the adversary does not have control over any DNS zone(s) relevant
to the function of \SYSTEM{}.

If the adversary does not alter the Resource Identifier, the compromised
resource will fail integrity validation during the NAH Validation step.

If the adversary does alter the Resource Identifier, there are two
possibilities: a) the new Resource Identifier \textit{does not} exist in the DNS
zone, in which case \SYSTEM{} will fail to resolve the NAH, hence the NAH
Validation step will fail; b) the new Resource Identifier \textit{does} exist in
the DNS zone, therefore the ``new'' RI must be pointing to a different file's
hash. Unless the adversary's goal is to swap one or more files protected by
\SYSTEM{} and a particular DNS zone with another file also protected by
\SYSTEM{} and in that same zone, the NAH Validation step will fail. For the
aforementioned ``swap'' to work, the adversary would be required to both change
the RI and also offer to the victim the \SYSTEM{} protected file the ``new'' RI
corresponds to, which shrinks the attack surface significantly.

\subsubsection{Compromised Authoritative Hash}

We consider the case where an adversary can completely control the victim DNS
zone(s) that allow \SYSTEM{} to function. Therefore, the adversary can return an
authoritative response of their choice to any DNS query.

In this case, the adversary does not have control over the victim's file
distribution mechanism (web page, file server, CDN, etc).

DNSSEC ensures the validity and authenticity of DNS responses. In order for the
adversary to control any relevant DNS zones, they must have access to the
authoritative DNS server and/or the appropriate DNSSEC keys.

Even if the adversary achieved this level of compromise, they do not have the
ability to deliver a malicious payload in this case. However, the adversary
could use control over the relevant DNS zones to cause denial-of-service attacks
against those attempting to download the resource by causing all NAH Validation
checks to fail. This is mitigated by \SYSTEM{} allowing the user to ``override''
its error states, similarly to Google Chrome's invalid HTTPS certificate warning
page allowing advanced users to pass through.

\subsubsection{Compromised Resource and Authoritative Hash}

We consider the case where an adversary can influence or even completely control
the victim's resource distribution mechanism (web page, file server, CDN, etc)
in any way. Additionally, the adversary can completely control the victim DNS
zone(s) that allow \SYSTEM{} to function. Therefore, the adversary can make the
user download a compromised resource and also return a (compromised) AH that
legally corresponds to said compromised resource.

\subsubsection{Determining the Origin Domain}

If an adversary manages to compromise a web page/server, they have two options.
They can mutate the resource directly, which would be observable via \SYSTEM{}.
They could also mutate the web/download page itself, replacing the anchor with a
malicious one that points to a compromised resource on the adversary's remote
system. This system could be configured with valid \SYSTEM{} DNS TXT records,
allowing the adversary to trick \SYSTEM{} into green lighting the resource
without complaint. Similarly, an adversary could redirect the user to an valid
and innocuous (but compromised) page that very quickly redirects the user again
to the compromised resource with the goal of tricking \SYSTEM{}.

In order to prevent such implementation-level attacks, we make a distinction
between the domain that the hyperlink containing the desired resource references
and the OD---or the domain of the document within which said hyperlink exists.
The extension must be implemented such that the OD is resolved as early as
possible in the page loading process. The scope of the OD is at the tab level,
meaning there is one OD determined for each open browser tab. Once determined
for a tab, the OD should not be recalculated for some period of time. If the
browser is navigated within this time period, the user will be asked to verify
that the OD is what they expect it to be (should be a familiar URL).

We implement this mitigation using chrome Download API's DownloadItem::referrer
property as the OD. We catch redirection attacks by assuming any page that
begins a download in under 3 seconds is suspicious and requires affirmation by
the user.

\TODO{Figuring out OD for FTP is really easy, though}

\subsection{Real-World Resource Corruption Detection with Google Chrome and HotCRP}

\TODO{It also seems from ACME that HTTP challenges are good enough of a proof to
issue TLS certificates, so why not good enough for checksums? Threat model of
ACME thoroughly goes through this~\cite{draft-ACME}.}

\TODO{We carefully evaluate the security, performance,
and practicality of \SYSTEM{} through implementing extensions in both a web
browser (Chrome) and FTP client (FileZilla); implementations are tested versus
common resource integrity violations. We find that \SYSTEM{} is more effective
than existing solutions, detects a wide variety of real-world integrity errors
across a diverse set of platforms, and is a scalable and immediately deployable
solution.}

\subsection{Deployment, Scalability, and Robustness}

\TODO{Discuss envisioned deployment strategies for resource providers.}

\TODO{Can this be scaled? Yes it can. What are the practical limits? EDNS0 means
it is not DNS size, though packet fragmentation is still a concern. How about
max record length? Maximum number of records? A service could have thousands or
millions of files it serves! Can DNS handle that? DHT failover is still a
solution anyway.}

\subsection{Performance Overhead}

\TODO{Additional Download Latency, Additional Network Load, Runtime overhead,
etc. All nixed.}

\input{6_related}
\section{Conclusion} \label{sec:conclusion}

Downloading resources over the internet is indeed a risky endeavor. Resource
integrity attacks, and Supply Chain Attacks more broadly, are becoming more
frequent and their impact more widely felt. This paper shows that the de facto
standard for addressing resource integrity risk---the use of \emph{checksums}
coupled with a secure transport layer---is an insufficient and often ineffective
solution. We propose a novel resource validation scheme meant as a complete
replacement for checksum based approaches: \SYSTEM{}, which automates the
tedious parts of verification to eliminate user apathy while leveraging
highly-available authenticated distributed systems to ensure resources and
checksums are not co-hosted. Further, we demonstrate the effectiveness and
practicality of our approach versus resource integrity attacks in real-world
systems.

The results of our evaluation show that \SYSTEM{} is more effective than
checksums and other existing solutions at mitigating resource integrity attacks
against arbitrary resources on the internet. Further, we show \SYSTEM{} detects
a wide variety of real-world integrity errors across a diverse set of platforms.
\SYSTEM{} is both scalable and immediately deployable for organizations that
secure their DNS zone(s) with DNSSEC.

We make our DNS implementation of \SYSTEM{} available open source so that others
can extend it or compare to it. Our hope is that this work motivates further
exploration of resource integrity Supply Chain Attack mitigation
methods.~\footnote{The \SYSTEM{} Chrome extension is available at
https://tinyurl.com/dnschk-actual}.


\clearpage
\printbibliography 
\end{document}
