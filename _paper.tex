%\documentclass[10pt,preprint,nocopyrightspace,nonatbib]{sigplanconf}
%\documentclass[9pt,preprint]{sig-alternate-no-permission}
%\documentclass[9pt,preprint]{sig-alternate}
%\documentclass[letterpaper,twocolumn,10pt]{article}
%\usepackage{usenix}
%\documentclass[pageno]{jpaper}
%\documentclass[10pt,preprint]{sigplanconf}
\documentclass[sigplan,10pt,anonymous]{acmart}\settopmatter{printfolios=true}
\usepackage{mathptmx} % This is Times font

\setcopyright{none}             %% For review submission
%\conferenceinfo{SOSP'15}{October 4--7, 2015, Monterey, CA}
%\copyrightyear{2015}
% The following \documentclass options may be useful:

% preprint      Remove this option only once the paper is in final form.
% 10pt          To set in 10-point type instead of 9-point.
% 11pt          To set in 11-point type instead of 9-point.
% authoryear    To obtain author/year citation style instead of numeric.

\usepackage{minted}
\usepackage{epsfig}
%\usepackage[utf8x]{inputenc}
\usepackage{algorithm}
\usepackage{amsmath, amssymb}
\usepackage[noend]{algpseudocode}
\usepackage{enumitem}      % adjust spacing in enums
%\usepackage{subfig}
\usepackage{caption}
\usepackage{subcaption}
\usepackage{multirow}
\usepackage{rotating}
\usepackage{wrapfig}
\usepackage{tabu}

\let\bibhang\relax
\let\citename\relax
\let\bibfont\relax
\let\citeauthor\relax
\let\Citeauthor\relax
\let\citefullauthor\relax
\let\citetext\relax
\let\defcitealias\relax
\let\citet\relax
\let\citep\relax
\let\Citep\relax
\let\Citealt\relax
\let\citealt\relax
\let\citealp\relax
\let\Citealp\relax
\let\Citet\relax

\expandafter\let\csname ver@natbib.sty\endcsname\relax

\DeclareCaptionFormat{subfig}{\figurename~#1#2#3}
\DeclareCaptionSubType*{figure}
\captionsetup[subfigure]{format=subfig,labelsep=colon,labelformat=simple}

\usepackage[natbib=true,backend=bibtex,firstinits=true,style=numeric-comp,sorting=nyt,defernumbers,maxnames=99,maxcitenames=99]{biblatex}
\usepackage{balance}
\usepackage{adjustbox}

\usepackage{pgfplots}
% options for pgfplots
\pgfplotsset{compat=1.8,compat/show suggested version=false}
\usetikzlibrary{plotmarks}
\usetikzlibrary{calc}
%\pgfplotsset{compat=newest}
\pgfplotsset{
   /pgfplots/bar  cycle  list/.style={/pgfplots/cycle  list={%
        {black,fill=black!30!white,mark=none},%
        {black,fill=red!30!white,mark=none},%
        {black,fill=green!30!white,mark=none},%
        {black,fill=yellow!30!white,mark=none},%
        {black,fill=brown!30!white,mark=none},%
     }
   },
}
% begin of externalization
\usetikzlibrary{external}
\tikzexternalize[prefix=out/]
\tikzexternalize
% don't externalize todonotes
%\makeatletter
%\renewcommand{\todo}[2][]{\tikzexternaldisable\@todo[#1]{#2}\tikzexternalenable}
%\makeatother
% end of externalization
\usetikzlibrary{patterns}
\usepgfplotslibrary{groupplots}
\pgfplotsset{
every axis label/.append style={font=\small},
tick label style={font=\small},
}
% options for paragraphs and lists
\setlist{noitemsep,topsep=0pt}
% options for biblatex
\bibliography{refs}
%\renewcommand{\bibfont}{\footnotesize}


\usepackage{fancyhdr}

% Ensure letter paper
\pdfpagewidth=8.5in
\pdfpageheight=11in

%%%%%%%%%%%---SETME-----%%%%%%%%%%%%%
\newcommand{\microsubmissionnumber}{440}
\newcommand{\asplossubmissionnumber}{440}
%%%%%%%%%%%%%%%%%%%%%%%%%%%%%%%%%%%%

\fancypagestyle{firstpage}{
%   \fancyhf{}
% \setlength{\headheight}{50pt}
% \renewcommand{\headrulewidth}{0pt}
%   \fancyhead[C]{\normalsize{ASPLOS 2018 Submission
%       \textbf{\#\asplossubmissionnumber} -- Confidential Draft -- Do NOT Distribute!!}} 
%   \pagenumbering{arabic}
}

\pagenumbering{arabic}


\setlength{\itemsep}{0pt}
%\setlength{\topsep}{0pt}
%\setlength{\partopsep}{0pt}
%\setlength{\parsep}{1pt}
%\setlength{\parskip}{1pt}
\setlength{\abovecaptionskip}{1pt plus 2pt minus 2pt}
\setlength{\textfloatsep}{5pt}
%\setlength{\bibitemsep}{0pt}

\sloppy

\newif{\ifanonymous}
\anonymoustrue

%\newcommand{\comment}[[1]{\textcolor{red}{#1}}
\newcommand{\myworries}[1]{\textcolor{red}{#1}}
\newcommand{\cutout}[1]{}
\newcommand{\smallcaption}[1]{\caption[#1]{{\protect\small \protect\bf #1}}}
\newcommand{\dids}{{\sc dids}}

\graphicspath{{./figs/}}
\date{}

\algnewcommand{\LineComment}[1]{\(\triangleright\) #1}

% some useful shortcuts
\newcommand{\ie}{\textit{i.e., }}
\newcommand{\eg}{\textit{e.g., }}
\newcommand{\CC}{C\nolinebreak\hspace{-.05em}\raisebox{.5ex}{\tiny\bf +}\nolinebreak\hspace{-.10em}\raisebox{.5ex}{\tiny\bf +}}

% units for results
\newcommand{\us}{\,$\mu$s}
\newcommand{\ms}{\,ms}
\newcommand{\KB}{\,KB}
\newcommand{\MB}{\,MB}
\newcommand{\GB}{\,GB}
\newcommand{\MHz}{\,MHz}
\newcommand{\GHz}{\,GHz}

% new latex commands:
%   Remove long section
\newcommand{\PUNT}[1]{}
\newcommand{\TABLETWO}[1]{}
%   Label work to be done
\definecolor{gray}{gray}{0.75}
\newcommand{\TODO}[1]{\textcolor{gray}{\textbf{\ [TODO:\ #1]\ }}}
\newcommand{\TR}[1]{#1}
%\newcommand{\TR}[1]{}
%\newcommand{\TODO}[1]{}
\newcommand{\FIX}[1] {\textcolor{red}{\textbf{\ [FIX:\ #1]\ }}}
%   Referencing various pieces of the document:
\newcommand{\figref}[1]{Fig.~\ref{fig:#1}}
\newcommand{\figsref}[2]{Figures~\ref{fig:#1} and~\ref{fig:#2}}
\newcommand{\figrref}[2]{Figures~\ref{fig:#1}--\ref{fig:#2}}
\newcommand{\secref}[1]{Section~\ref{sec:#1}}
\newcommand{\secsref}[2]{Sections~\ref{sec:#1} and~\ref{sec:#2}}
\newcommand{\eqnref}[1]{Eqn.~\ref{eqn:#1}}
\newcommand{\eqnsref}[2]{Equations~\ref{eqn:#1} and~\ref{eqn:#2}}
\newcommand{\eqnrref}[2]{Equations~\ref{eqn:#1}--\ref{eqn:#2}}
\newcommand{\insref}[1]{Instruction~\ref{ins:#1}}
\newcommand{\tblref}[1]{Table~\ref{tbl:#1}}
\newcommand{\appref}[1]{Appendix~\ref{app:#1}}

\newcommand{\algoref}[1]{Algorithm~\ref{algo:#1}}

% Custom hyphenation rules

%\DeclareMathOperator{\minimize}{minimize}
%\DeclareMathOperator{\st}{s.t.}
%\DeclareMathOperator*{\argmin}{arg\,min}
%\DeclareMathOperator*{\argmax}{arg\,max}
\newcommand{\argmin}{\arg\!\min}
\newcommand{\argmax}{\arg\!\max}
\newcommand{\minimize}{minimize}
\newcommand{\optimize}{optimize}
\newcommand{\ceil}[1]{\lceil #1 \rceil}
\newcommand{\floor}[1]{\lfloor #1 \rfloor}
\newcommand{\st}{s.t.}

\newcommand{\SYSTEM}{DNSCHK}

\pdfstringdefDisableCommands{
    \def\\{}
    \def\unskip{}
    \def\texttt#1{<#1>}
}

%-------------------------------------------------------------------------
\begin{document}

\date{}
\title[\SYSTEM{}]{\SYSTEM{}: \TODO{Needs After Colon Title}}

\author{
% Not used?
}

\begin{abstract}

  Downloading resources over the internet comes with many risks,
  including the chance that a malicious actor has replaced the
  resource you think you are accessing with a compromised version. The
  current standard for addressing this risk is the use of
  \emph{checksums} coupled with a secure transport layer: users
  download a resource and compare its checksum with a posted checksum
  from the developers to ensure a match. Among the many problems with
  the current use of checksums are (1) user apathy---for most users,
  hand-calculating the checksum and comparing to the published version
  are too tedious---and (2) co-hosting resources and checksums---a
  malicious actor who compromises a resource can trivially compromise
  a checksum a hosted using the same infrastructure. In this paper we
  propose \SYSTEM{}, a novel resource validation scheme meant as a
  complete replacement for current checksum based approaches.
  \SYSTEM{} automates the tedious parts of verification to eliminate
  user apathy while leveraging highly-available distributed systems to
  separate resources from digests, making these systems and their end
  users more resilient to resource integrity attacks.  We carefully
  evaluate the security, performance, and practicality of \SYSTEM{}
  through implementing extensions in both a web browser (Chrome) and
  FTP client (FileZilla); implementations are tested versus common
  resource integrity violations. We find that \SYSTEM{} is more
  effective than existing solutions, detects a wide variety of
  real-world integrity errors across a diverse set of platforms, and
  is a scalable and immediately deployable solution.

\end{abstract}


\maketitle
\thispagestyle{firstpage}
\pagestyle{plain}



%\vskip -100pt


%\category{C.1.3}{Other Architectural Styles}{Adaptable architectures}
%\category{I.2.8}{ Problem Solving, Control Methods, and Search}{Heuristic methods}
%\category{D.4.8}{Performance}{Measurements}
%\terms{Performance, Design, Experimentation}
%\keywords{Adaptive Systems, Power Management, Decision-tree}


\section{Introduction} \label{sec:introduction}

Using the internet to receive data is a remarkably simple and painless process
for application developers and end users alike. When using a browser such as
Google Chrome, this process can be summarized as: a user requests a server
resource at some URL, the server responds with the desired resource, and then
the browser completes downloading the resource.

Unfortunately, downloading content over the internet can be risky.

Supply Chain Attacks (SCA) are the compromise of software source code via cyber
attack, insider threat, or other attack on one or more phases of the development
and deployment life cycle. These attacks are made possible due to proximity and
have the goal of infecting and exploiting one or more victims---usually the
software company's customer base. \SYSTEM{} only protects against SCAs that
occur after the Authoritative Hash (AH) is calculated. AH calculation is
necessarily more likely to occur later in the software development life cycle or
very early in the deployment process. If an attacker is able to execute a
successful SCA before the AH is calculated, \SYSTEM{} would propagate the
compromised Authoritative Hash.

\TODO{(include table of supply chain stages that \SYSTEM{} is effective in)}

Although early Supply Chain Attacks are devastating, they are not the only
popular form of the attack. Many devastating supply chain attacks occur late in
the software development and deployment process, including \TODO{(choose three examples)}. Further, \TODO{(popular attacks at various levels, including CDNs)}.

Discuss ``free," \ie no interface changes, no addition to resource download
time, no additional burden on the end user (qualified statement).~\cite{DNSSEC}

In summary, our primary contributions are:

\begin{itemize}

  \item We propose a novel practical defense against receiving malicious,
  corrupted, or compromised resources over the internet. Contrasted with current
  solutions, our defense requires no source code or infrastructure changes at
  any level other than DNS, does not employ unreliable heuristics, does not
  interfere with other software or extensions that also handle resource
  downloads, and can be transparently deployed without adding to the
  \textit{fragility} of DNSSEC-enabled systems; it protects end users whose
  software implements \SYSTEM{} while remaining unnoticeable to users of whose
  software does not.

  \item We present our prototype \SYSTEM{} implementations for Google Chrome and
  FileZilla and demonstrate its effectiveness in automatically and transparently
  mitigating the accidental consumption of compromised resources from a
  compromised server hosting a compromised web portal. To the best of our
  knowledge, this is the \emph{first} system providing such capabilities with
  little implementation cost and at no cost to the end user. Further, we release
  the \SYSTEM{} solution to the community as open source software\footnote{The
  Chrome extension is available at https://tinyurl.com/dnschk-actual}.

  \item We carefully and extensively evaluate the security, scalability, and
  performance of our automated defense against resource corruption to
  demonstrate the effectiveness and high practicality of the \SYSTEM{} approach.
  Specifically, we find no obstacles to efficient scalability given choice of
  distributed system and no performance overhead compared to downloads without
  \SYSTEM{}. We further provide a publicly accessible demonstration of
  \SYSTEM{}'s utility via a patched HotCRP instance\footnote{The patched HotCRP
  instance is available at https://tinyurl.com/dnschk-hotcrp}.

\end{itemize}

\section{Specification} \label{sec:specification}

\SYSTEM{} acts as a translation layer sitting between the drive and the
operating system. It provides confidentiality and integrity guarantees while
minimizing performance loss due to metadata management overhead. \SYSTEM{}
accomplishes this by leveraging the speed of stream ciphers over the AES block
cipher and taking advantage of the append-mostly nature of Log-structured
Filesystems (LFS) and modern Flash Translation Layers (FTL)~\cite{SSD}.

\section{\SYSTEM{} Implementations} \label{sec:implementations}

\TODO{todo}

\section{Evaluation} \label{sec:evaluation}

The primary goal of any \SYSTEM{} implementation is to alert end-users when the
resource they have downloaded is something other than what they were expecting.
In this section, we evaluate our approach by first assessing the threat model
\SYSTEM{} addresses, followed by an examination of our proof-of-concept Google
Chrome extensions, \DNSSYS{} and \DHTSYS{}. We then test our implementations
versus a real-world deployment of HotCRP and/or a random sampling of papers
published in previous \CONFERENCE{} proceedings. Finally, we evaluate the
obstacles to scalability and potential performance overhead of our extensions.

\subsection{The Threat Model}

\subsubsection{Compromised Resource}

We consider the case where an attacker can influence or even completely control
the victim's resource distribution mechanism (web page, file server, CDN, etc)
in any way. In this context, the attacker can trick the user into downloading a
compromised resource of the attacker's choice. This attack can be accomplished
by compromising the resource on a victim system or tricking the user into
downloading a compromised resource on the attacker's remote system.

In either case, the attacker does not have control over the backend system
responsible for mapping RIs to ACs relevant to the function of \SYSTEM{}.

If the attacker does not alter the RI, the compromised resource will fail
integrity validation during the NAC Validation step.

If the attacker does alter the RI, there are two possibilities: 1) the new RI
\textit{does not} exist in the backend, in which case \SYSTEM{} will fail to
resolve the NAC, hence the NAC Validation step will fail; 2) the ``compromised''
RI \textit{does} exist in the backend, therefore the RI must be pointing to a
different resource's checksum.

In the first case, there are two further possibilities: a) NAC validation fails
and \SYSTEM{} \emph{is not} in strict mode, so a ``neutral'' judgement is
rendered; b) NAC validation fails and \SYSTEM{} \emph{is} in strict mode, so an
``unsafe'' judgement is rendered, warning end users that the resource is likely
compromised.

In the second case, unless the attacker's goal is to swap one or more resources
protected by \SYSTEM{} and a particular backend with another resource also
protected by \SYSTEM{} and the same backend, the NAC Validation step will fail.
For such a ``swap'' to work, the attacker would be required to both change the
RI and also offer to the victim the \SYSTEM{}-protected resource the
``compromised'' RI corresponds to, which shrinks the attack surface here
significantly.

\subsubsection{Compromised Authoritative Checksum}

We consider the case where an attacker can completely control the highly
available backend that allows \SYSTEM{} to function. In this context, the
attacker can return an authoritative response of their choice to any query.

In this case, the attacker does not have control over the victim's resource
distribution mechanism (web page, file server, CDN, etc).

Even if the attacker achieved this egregious level of compromise, they do not
have the ability to deliver a malicious payload in this case. However, the
attacker could use control over the relevant backends to cause denial-of-service
style attacks against those attempting to download the resource by causing all
NAC Validation checks to fail. This is mitigated by \SYSTEM{} allowing the user
to ``override'' its error states; \ie \SYSTEM{} does not mutate or quarantine
downloaded resources. See \secref{discussion} for further discussion on
limitations due to the Chrome/WebExtensions API.

\subsubsection{Compromised Resource and Authoritative Checksum}

We consider the case where an attacker can influence or even completely control
the victim's resource distribution mechanism (web page, file server, CDN, etc)
in any way. Additionally, the attacker can completely control the victim highly
available backend that allows \SYSTEM{} to function. Therefore, the attacker can
make the user download a compromised resource and also return a (compromised) AC
that legally corresponds to said compromised resource.

\subsection{Real-World Resource Corruption Detection}

To further evaluate the effectiveness of our mitigation, we test \DNSSYS{} and
\DHTSYS{}---our proof-of-concept \SYSTEM{} Chrome extension
implementations---against a series of common real-world resource integrity
violations. The impetus behind any such resource integrity SCA is to have the
resource pass through undetected by abusing the trust between client and
provider with the hope that an unsuspecting user will interact with it.

We show that the \SYSTEM{} approach is more effective than existing approaches
at detecting integrity violations in arbitrary resources on the internet; this
is especially evident when \DNSSYS{} and \DHTSYS{} are compared to the de facto
standard: checksums.

\subsubsection{\DNSSYS{}}

To empirically evaluate \DNSSYS{}, we launch a heavily modified version of the
popular open source research submission and peer review software, \emph{HotCRP}
(version 2.102). Our modifications allow anyone visiting the site to
interactively corrupt submissions and manipulate relevant DNS entries at will.

For our evaluation, we upload 10 different \CONFERENCE{} PDFs to our HotCRP
instance. Upon their upload, HotCRP calculated and displayed the unique checksum
(a SHA-256 digest) of each PDF. After each PDF is uploaded, we immediately
download it and manually calculate a local checksum, matching each to the
checksum displayed by the HotCRP software. Next, we utilize the custom
functionality we added to our HotCRP instance to populate our DNS backend with
each file's current "original" checksum. Each checksum is considered an
Authoritative Checksum (AC) mapped to a Resource Identifier (RI) corresponding
to the uploaded PDF item.

After installing \DNSSYS{} into our Google Chrome browser, we again download
each file. For each observed download, \DNSSYS{} reported a ``safe'' judgement
as expected. We then utilize the custom functionality we added to our HotCRP
instance to add junk data onto the end of each of the uploaded PDFs, corrupting
them. We also modified HotCRP so that it updated the displayed checksums to
match their now-corrupted counterparts.

Once again, we download each file and calculate a local checksum. \DNSSYS{}
reported an ``unsafe'' judgement (a true positive) for each corrupted PDF file,
as expected. Calculating the local checksum and checking it against the value
reported by our HotCRP instance leads to a match (a false negative; \ie the
result of co-location) as expected.

We ran this experiment three times and observed consistent results.

Finally, we implement a ``redirection'' attack where, when clicking the link
to download the PDF document from HotCRP, users were forced to navigate to a
``compromised'' PHP script on an adjacent server that very quickly redirected
clients several times before triggering the download of a corrupted version of
the original HotCRP-hosted resource. Our implementation correctly flagged this
download as suspicious once the download began, successfully warning the user.

\subsubsection{\DHTSYS{}}

To evaluate \DHTSYS{}, we connect to the authenticated global Ring OpenDHT
network. Since the OpenDHT software is implemented in C++, it could not be
included directly in a JavaScript plugin. Therefore, for our proof-of-concept
implementation, we set up a local HTTP REST server wrapping the C++
implementation of OpenDHT. Our OpenDHT REST server provides an interface
consistent with the one expected by \DNSSYS{} (\ie Google DNS's REST API),
allowing for code reuse (\eg redirection protection) between our two
implementations. See \secref{availability}, where we make these implementations
available open source for community consideration.

For our evaluation, we manually calculate a local checksum (\ie an AC) and an
RI for 10 different \CONFERENCE{} PDFs. For the OD here we use a static locally
resolved domain name corresponding to a location on a local server. We manually
store these RI-to-AC mappings as key-value pairs on the Kademlia-based OpenDHT
network.

After installing \DHTSYS{} into our Google Chrome browser, we download each
file from our local server. For each observed download, \DHTSYS{} reported a
``safe'' judgement as expected. We then store randomly generated checksums as
replacement RI-to-AC mappings in the Ring OpenDHT network corresponding to these
10 PDFs such that the ACs purposely would not match the NACs generated by
\DHTSYS{}, simulating file corruption by an attacker.

Once again, we download each file and calculate a local checksum. \DHTSYS{}
reported an ``unsafe'' judgement for each corrupted PDF file, as expected.

\subsection{Obstacles to Scalability, Deployment}

As \SYSTEM{} is predicated on a distributed authenticated highly-available
backend and requires no application/frontend source code changes to function, we
conclude that the scalability of \SYSTEM{} can be reduced to the scalability of
its backend. We are aware of no other obstacles to scalability beyond those
inherited from the underlying backend system.

In respect to DNS specifically, packet fragmentation can be a concern for high
performance networks~\cite{EDNS}, but this is an artifact of DNSSEC and related
protocols rather than \SYSTEM{}~\cite{DNSSEC}. Further, we are aware of no
practical limits or protocol-based restrictions on the scalability of a backend
file itself or its sub-zones. A service can host tens of thousands of resource
records in their backend file~\cite{DNS1, DNS2}.

With the HotCRP demo, the totality of our resource deployment scheme consisted
of the addition of a new TXT entry to our backend file---accomplished via API
call to Google DNS---during HotCRP's paper submission process. This new TXT
entry consisted of a mapping between a RI and its corresponding AC.

We find a DNS record addition or update during the resource deployment process
to be simple enough for service administrators to implement and presents no
significant burden to deployment outside of DNS API integration into a
development team or other entity's software deployment toolchain. For reference,
we implemented the functionality that automatically adds (and updates) the DNS
records mapping the ACs and RIs of papers uploaded to our HotCRP instance in
under 10 lines of JavaScript.

We note that, in the case where an entity's content distribution mechanism
relies on, for instance, a mirroring service, third-party CDN, et cetera
\emph{that randomly or disparately mangles resource URL paths}, our
proof-of-concept implementations currently require each "mangled" RI permutation
representing a resource to be added to the backend, even if they all represent
the same resource by a different name/path. This issue and its solutions are
discussed further in the context of URNs in \secref{discussion}.

\subsection{Performance Overhead}

While evaluating our \SYSTEM{} implementations, we observe no additional
network load or CPU usage with the extension loaded into Chrome.
Measurements were taken using the Chrome developer tools. Intuitively, this
makes sense since our \SYSTEM{} implementations make at most two queries to the
backend before rendering a judgement. Hence, we find that \SYSTEM{} introduces
no additional performance overhead. Further, as our implementations
of \SYSTEM{} do not interrupt or manipulate resources as they are being
downloaded, there is no additional download latency introduced by \SYSTEM{}.

\subsection{Attacking OD Resolution in the Browser}

While determining ODs, a clever attacker might attempt to fool this process by
redirecting users one or more times before triggering a direct download of a
compromised resource. The redirects would allow an attacker to completely
manipulate the OD, with the ability to trick an unsuspecting user into
downloading a compromised resource with valid entries in the backend system.

We mitigate this threat by leveraging JavaScript's document-wide \emph{trusted
event}~\cite{TrustedEvents} delegation capability. Specifically, when a tab
navigation event is observed, the tab is flagged \texttt{suspicious} by default
and the determined OD is not updated (\ie it remains at its previous value). If
the user interacts with the tab (\ie a trusted click or key press event is
observed after navigation completes), the \texttt{suspicious} flag is cleared
and the OD is updated. If another tab navigation event occurs without user
interaction first (\eg a quick redirect), this process repeats recursively. If a
download is observed coming from a tab flagged \texttt{suspicious}, the user is
warned about the suspicious circumstances similarly to receiving an ``unsafe''
judgement.

While this mitigates the attack as described, it has the side effect of
potentially generating false positive warnings when 1) the user is redirected to
a legitimate website---such as a download mirroring service---that automatically
triggers a download after some amount of time when also 2) the user does not
interact with the page at all before the download begins. We argue such cases
are non-average and the tradeoff here is worthy.

\section{Related Work} \label{sec:related}

\TODO{(adapt related work collected from wiki)}

\TODO{(cite some usenix/oakland papers)}

\section{Conclusion} \label{sec:conclusion}

Downloading resources over the internet is indeed a risky endeavor. Resource
integrity attacks, and Supply Chain Attacks more broadly, are becoming more
frequent and their impact more widely felt. This paper shows that the de facto
standard for addressing resource integrity risk---the use of \emph{checksums}
coupled with a secure transport layer---is an insufficient and often ineffective
solution. We propose a novel resource validation approach meant as a complete
replacement for checksum based approaches: \SYSTEM{}, which automates the
tedious parts of verification to eliminate user apathy while leveraging
highly-available authenticated distributed systems to ensure resources and
checksums are not co-hosted. Further, we demonstrate the effectiveness and
practicality of our approach versus resource integrity attacks in a real-world
system.

The results of our evaluation show that our approach is more effective than
checksums and other attempts at mitigating resource integrity attacks
against arbitrary resources on the internet. Further, we show \DNSSYS{} is
capable of detecting a wide variety of real-world integrity errors,
significantly raising the bar for the attacker. \DNSSYS{}, as it is backed by
DNS, is immediately deployable at scale for entities that secure their DNS
zone(s) with DNSSEC.


\clearpage
\printbibliography 
\end{document}
