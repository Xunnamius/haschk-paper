\section{Conclusion} \label{sec:conclusion}

Downloading resources over the internet is indeed a risky endeavor. Resource
integrity attacks, and Supply Chain Attacks more broadly, are becoming more
frequent and their impact more widely felt. This paper shows that the de facto
standard for addressing resource integrity risk---the use of \emph{checksums}
coupled with a secure transport layer---is an insufficient and often ineffective
solution. We propose a novel practical resource validation approach meant as a
complete replacement for checksum based approaches: \SYSTEM{}, which automates
the tedious parts of verification to eliminate user apathy while leveraging
highly-available authenticated distributed systems to ensure resources and
checksums are not co-hosted. Further, we demonstrate the effectiveness and
practicality of our approach versus resource integrity attacks in a real-world
system.

The results of our evaluation show that our approach is more effective than
checksums and other attempts at mitigating resource integrity attacks for
arbitrary resources on the internet. Further, we show \DNSSYS{} and \DHTSYS{}
are capable of detecting a wide variety of real-world integrity errors,
significantly raising the bar for the attacker. \DNSSYS{}, as it is backed by
DNS, is deployable at scale for entities that already maintain a web presence;
this can be done without fear of adversely affecting the user experience of
non-compliant clients.

\PUNT{\subsection{Future Work}

\subsubsection{Merkle Trees and Early Resource Validation}

Using Merkle trees~\cite{Merkle} instead of pure cryptographic hashing functions
for resource validation would enable partial verification of large files. For
example, suppose we are downloading a 10TiB resource and it is compromised. By
calculating a Merkle tree beforehand, we do not have to wait for the resource to
finish downloading before we render a failing judgment. This partial
verification has the potential to save the user a significant amount of time,
though using Merkle trees for resource integrity validation over the internet is
decidedly not-trivial~\cite{Merkle-HTTP}.

For a production example of Merkle tree based resource integrity validation, we
can look to the so-called \emph{Tiger tree hash}~\cite{TTH, Merkle} (TTH)
construction. The TTH, a Merkle tree implementation, is built on the Tiger
cryptographic hashing function. Merkle trees and TTHs are well-studied and
widely deployed constructions capable of supporting ``partial verification'' of
resources as they are downloaded. Tiger tree hashes in particular are popular
among several large P2P file sharing applications such as WireShare
(LimeWire)~\cite{LimeWire}. Of course, a solution need not be tightly coupled to
the Tiger cryptographic hashing function. The high-speed BLAKE2, SHA2, or SHA3
cryptographic hashing functions would perform just as well, if not better.}
