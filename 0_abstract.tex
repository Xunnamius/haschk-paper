\begin{abstract}

Downloading resources over the internet comes with many risks, including the
chance that the resource has been corrupted, or that an attacker has replaced
your desired resource with a compromised version. The de facto standard for
addressing this risk is the use of \emph{checksums} coupled with a secure
transport layer; users download a resource, compute its checksum, and compare
that with an authoritative checksum. Problems with this include (1) \emph{user
apathy}---for most users, calculating and validating the checksum is too
tedious; and (2) \emph{co-hosting}---an attacker who compromises a resource can
trivially compromise a checksum hosted on the same system. The co-hosting
problem remains despite advancements in tools that automate checksum
verification and generation. In this paper we propose \emph{\SYSTEM{}}, a
resource validation approach expanding on de facto checksum-based integrity
protections to defeat co-hosting while automating the tedious parts of checksum
verification. We evaluate the security, practicality, and ease of deployment of
our approach through two DNS and DHT based implementations in Google Chrome;
implementations are tested versus common resource integrity violations. We find
our approach is more effective than existing mitigation methods, significantly
raises the bar for the attacker, and is deployable at scale.

\end{abstract}
