\begin{abstract}
  \TODO{This abstract is way to long. There is a lot of good stuff in
    here, but much of it needs to be moved to the introduction.  The
    goal should be to try to get the whole thing into 6 sentences.
    Doing so will be hard, but that is sort of the point---the process
    forces you to think about what is really key in your work. I
    always start by answering the following: (1) What problem are we
    trying to solve? (2) What has prior work done in this area? (3)
    Why is it not sufficient to solve the problem? (4) What are we
    proposing as an alternative? (5) How did we evaluate our approach?
    (6) What is its potential impact?  Note that in the abstract, we
    answer each of those questions with a sentence.  In the
    introduction, each of those sentences gets blown into a paragraph
    (where you can use a lot of the details in the original abstract).
    Then in the rest of the paper, each one of those paragraphs gets
    blown into a whole subsection.  \\ \\
    Here is my attempt to distill the current abstract into six
    sentences, each of which answers a corresponding question above (some answers I am not sure of, so we need to iterate on this):\\ \\
    Downloading resources over the internet comes with many risks,
    including the chance that a malicious actor has replaced the
    resource you think you are accessing with a compromised version.
    The current standard for addressing this risk is the use of
    \emph{checksums} coupled with a secure transport layer: users
    download a resource and compare its checksum with a posted
    checksum from the developers to ensure a match.  Among the many
    problems with the current use of checksums are: (1) user
    apathy---for most users hand-calculating the checksum and
    comparing to the published version are too tedious---and (2) a
    malicious actor who compromised the resource could often easily
    compromise the checksum as well.  In this paper we propose
    \SYSTEM{}, a novel resource validation scheme---meant as a
    complete replacement for current checksum based approaches---that
    automates the tedious parts of verification to eliminate user
    apathy, while leveraging \emph{<I am not sure what to put here>}
    to separate the resource from its authentication code. We evaluate
    \SYSTEM{} by implementing extensions to both a web browser
    (Chrome) and FTP client (FileZilla) while \emph{<something about
      the backend>}.  \emph{<Need some sentence on the
      impact here---in a straight systems paper I would talk about performance win or energy savings, etc.>} \\ \\
    Note that I had trouble in the last part of this abstract because,
    in the original abstract, it is not clear what the proposed
    solution actually is. Additionally, I think that the solution
    should be described in two parts: the user-facing part
    (implemented as the Chrome or FileZilla extension) and the
    back-end part (implemented on DNSSEC or a DHT).  If you are okay
    with the challenges I listed (user apathy and ease of compromising
    the resource and the checksum) then this structure is a nice
    mirror of the challenges and we should carry it through the
    paper.}


Downloading resources over the internet comes with considerable risk due to the
general inability of the end user to verify the integrity of the resource
they've received. An adversary could tamper with said resource in a variety of
ways: a) by compromising it en route to its destination (\eg via CDN), b) by
executing a successful supply chain or similar attack beforehand, or c) by
compromising the server(s) that physically hosts the resource, as was the case
with the 2016 breach of the HTTPS-protected Avast distribution servers that
provided downloads of the popular software suite CCleaner. The de facto standard
for addressing this risk is with the use of \textit{checksums} coupled with some
secure transport medium like TLS/HTTPS. Checksums in this context are
cryptographic digests generated by a cryptographic hashing function run over the
resource's file contents. They are supposedly hosted on a separate system than
the resources they protect and are used to verify a resource’s integrity to an
end user. Checksums are problematic for a whole host of reasons, the foremost
being that the clear majority of end users will not be burdened with manually
calculating them. And even if they did, said user must search for the
corresponding ``correct'' checksum to verify their calculation. Managing that,
if they then recognize the checksums are different, the user is then expected to
``do the right thing,'' whatever that happens to be in context.

With this research, we explore a novel method of verifying the integrity of
resources downloaded over the internet that is a complete replacement for
traditional checksums. We approach the problem with three key concerns in mind:
a) implementations provide security guarantees transparently without adding any
extra burden on the end user, \ie neither grappling with a new user interface
nor any additional labor is required during standard usage; b) configuring the
validation method is simple for service administrators and system operators to
integrate and deploy; and c) no new HTML or JavaScript language additions,
application source changes, or web server/infrastructure alterations are
necessary. Hence, we propose \SYSTEM{}, a novel automated resource validation
method that is transparent to end users and simple for administrators to deploy.
We implement \SYSTEM{} as a proof-of-concept Google Chrome extension as well as
a patch to the FileZilla FTP client. We evaluate the security, scalability, and
performance of the \SYSTEM{} approach and provide a publicly accessible
demonstration of its utility via a patched HotCRP instance.

\end{abstract}
