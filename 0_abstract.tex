\begin{abstract}

Downloading resources over the internet comes with considerable risk due to the
general inability of the end user to verify the integrity of the resource
they've received. An adversary could tamper with said resource in a variety of
ways: a) by compromising it en route to its destination, b) by executing a
successful supply chain or similar attack beforehand, or c) by compromising the
server that physically hosts the resource, as was the case with the 2016 breach
of the HTTPS-protected Avast distribution servers that provided downloads of the
popular software suite CCleaner. The de facto standard for addressing this risk
is with the use of \textit{checksums} coupled with some secure transport medium
like TLS/HTTPS. Checksums are fingerprints generated by cryptographic hashing
functions early in the build process, are supposedly hosted on a separate system
than a resource, and are used to verify that resource’s integrity to an end
user. Checksums are problematic for a whole host of reasons, the foremost being
the fact that the clear majority of end users will not be burdened with manually
calculating the checksum of a resource they've received. Even if they do, said
user must search for the corresponding ``correct" checksum to verify their
calculation. If they recognize the checksums are different, the user is then
expected to ``do the right thing" in context, whatever that happens to be.

With this research, we explore a novel method of verifying the integrity of
resources downloaded over the internet with two key concerns: a) it is
automatic, \ie no additional interaction is required from the end user during
standard usage and b) configuring the validation method is simple for service
administrators and system operators to integrate and deploy. Hence, we propose
\SYSTEM{}, a novel automated resource validation method that is transparent to
end users and simple for administrators to deploy. We implement \SYSTEM{} as a
proof-of-concept Google Chrome extension as well as a patch to the FileZilla FTP
client. We evaluate the security, scalability, and performance of the \SYSTEM{}
approach and provide a publicly accessible demonstration of its utility via a
patched HotCRP instance.

\end{abstract}
