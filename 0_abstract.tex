\begin{abstract}

Downloading resources over the internet comes with many risks, including the
chance that an attacker has replaced your desired resource with a compromised
version. The de facto standard for addressing this risk is the use of
\emph{checksums} coupled with a secure transport layer; users download a
resource, compute its checksum, and compare that with an authoritative checksum.
Problems with this include (1) \emph{user apathy}—for most users, calculating
and validating the checksum is too tedious; and (2) \emph{co-hosting}—an
attacker who compromises a resource can trivially compromise a checksum hosted
on the same system. In this paper we propose \emph{\SYSTEM{}}, a novel resource
validation scheme meant as a complete replacement for current checksum-based
approaches. \SYSTEM{} implementations automate the tedious parts of checksum
verification to sidestep user apathy while leveraging authenticated high
availability distributed systems to address co-hosting. We carefully evaluate
the security, performance, and practicality of our approach through
proof-of-concept implementations in Google Chrome and FileZilla; implementations
are tested versus common resource integrity violations. While not a panacea, we
find that our approach is more effective than existing mitigation methods,
significantly raises the bar for the attacker, and is scalable and immediately
deployable.

\end{abstract}
