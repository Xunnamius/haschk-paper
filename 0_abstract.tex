\begin{abstract}

Downloading files from the internet comes with considerable risk due to the
general inability of the end user to check the integrity of the file they
acquired. An adversary could tamper with the file en route to its destination or
even compromise the server that hosts the file, as was the case with the 2016
breach of the Avast distribution servers that provided downloads of the popular
software suite CCleaner. The traditional strategy for addressing this risk is
with the use of "checksums" or signatures generated by cryptographic hashing
functions to verify a file’s integrity. This process is problematic for a whole
host of reasons, especially the fact that the clear majority of end users cannot
be bothered to manually calculate a checksum, let alone find the corresponding
correct checksum via some web resource, compare the two, and then be expected to
"do the right thing".

With this research, we explore DNS-based methods of authenticating files
downloaded over the internet that are completely transparent to end users yet
dead simple for service administrators and system operators to comply with and
integrate. \TODO{Two? insights} Leveraging these insights, we propose \SYSTEM{},
a \TODO{description}. We implement \SYSTEM{} on a \SYSTEM{where?} and evaluate
its efficacy when integrated into multiple popular applications. We find that
\SYSTEM{} \TODO{what do we get?}.

\end{abstract}
