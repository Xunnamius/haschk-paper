\begin{abstract}

Downloading resources over the internet comes with many risks, including the
chance that a malicious actor has replaced the resource you think you are
accessing with a compromised version. The current standard for addressing this
risk is the use of \emph{checksums} coupled with a secure transport layer; users
download a resource and compare its checksum with a posted checksum from the
developers to ensure a match. Among the many problems with the current use of
checksums are (1) user apathy---for most users, hand-calculating the checksum
and comparing to the published version are too tedious; and (2) co-hosting
resources and checksums---a malicious actor who compromises a resource can
usually trivially compromise a checksum hosted on the same system. In this
paper, we propose \SYSTEM{}, a novel resource validation scheme meant as a
complete replacement for current checksum based approaches. \SYSTEM{} automates
the tedious parts of verification to eliminate user apathy while leveraging
highly-available distributed systems to separate resources from digests, making
these systems and their end users more resilient to resource integrity attacks.
We carefully evaluate the security, performance, and practicality of \SYSTEM{}
through implementing extensions in both a web browser (Chrome) and FTP client
(FileZilla); implementations are tested versus common resource integrity
violations. We find that \SYSTEM{} is more effective than existing solutions,
detects a wide variety of real-world integrity errors across a diverse set of
platforms, and is a scalable and immediately deployable solution.

\end{abstract}
