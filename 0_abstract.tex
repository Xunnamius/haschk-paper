\begin{abstract}

Downloading resources from the internet comes with considerable risk due to the
general inability of the end user to verify the integrity of the resource they
received. An adversary could tamper with the resource en route to its
destination or even compromise the server that hosts the resource, as was the
case with the 2016 breach of the Avast distribution servers that provided
downloads of the popular software suite CCleaner. The de facto standard method
for addressing this risk is with the use of \textit{checksums}—signatures
generated by cryptographic hashing functions to verify a resource’s
integrity—coupled with some secure transport medium like TLS/HTTPS. This method
is problematic for a whole host of reasons, the foremost being the fact that the
clear majority of end users will not be burdened with manually calculating a
resource's checksum. Even if they do, said user must search for the
corresponding ``correct" checksum to verify their calculation. If that
verification fails, the user is then expected to ``do the right thing" in
context.

With this research, we explore a novel method of verifying the integrity of
resources downloaded over the internet with two key concerns: a) it is
automatic, \ie no additional interaction is required from the end user during
standard usage and b) configuring the validation method is simple for service
administrators and system operators to integrate and deploy. Hence, we propose
\SYSTEM{}, a novel automated resource validation method that is transparent to
end users and simple for administrators to deploy. We implement \SYSTEM{} as a
proof-of-concept Google Chrome extension as well as a patch to the FileZilla FTP
client. We evaluate the security, scalability, and performance of the \SYSTEM{}
approach and provide a publicly accessible demonstration of its utility via a
patched HotCRP instance.

\end{abstract}
