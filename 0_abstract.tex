\begin{abstract}

Downloading resources from the internet comes with considerable risk due to the
general inability of the end user to verify the integrity of the resource they
received. An adversary could tamper with the resource en route to its
destination or even compromise the server that hosts the resource, as was the
case with the 2016 breach of the Avast distribution servers that provided
downloads of the popular software suite CCleaner. The traditional method for
addressing this risk is with the use of \textit{checksums}—signatures generated
by cryptographic hashing functions to verify a resource’s integrity—coupled with
some secure transport medium like TLS/HTTPS. These methods are problematic for a
whole host of reasons, the foremost being the fact that the clear majority of
end users will not be burdened with manually calculating a resource's checksum.
Even if they do, said user must search for the corresponding "correct" checksum
to verify their calculations against and, if that verification fails, is
expected to "do the right thing" in context.

With this research, we explore DNS-based methods of authenticating resources
downloaded over the internet that are completely transparent to end users yet
dead simple for service administrators and system operators to integrate and
deploy. \TODO{Two? insights} Leveraging these insights, we propose \SYSTEM{}, a
\TODO{description}. We implement \SYSTEM{} on a \SYSTEM{where?} and evaluate its
efficacy when integrated into multiple popular applications. We find that
\SYSTEM{} \TODO{what do we get?}.

\end{abstract}
